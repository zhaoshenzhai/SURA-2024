\documentclass{amsart}
\usepackage[hidelinks]{hyperref}                                            % links
\usepackage{pgfplots}\pgfplotsset{compat=1.18}                              % plots
\usepackage{amsfonts, amsmath, amssymb, amsthm}                             % basic maths commands
\usepackage{mathtools, mathrsfs}                                            % more maths commands
\usepackage{graphicx}                                                       % images and other graphics
\usepackage{geometry}                                                       % page layout
\usepackage{tikz, tikz-3dplot, tikzpagenodes}                               % maths figures
\usepackage{caption, subcaption}                                            % captions outside float
\usepackage{xcolor}                                                         % more colors
\usepackage{enumitem}                                                       % enumerate and itemize indents
\usepackage{nicematrix}                                                     % matrices and tables

\usetikzlibrary{matrix, positioning, patterns, decorations.markings, arrows, arrows.meta, backgrounds, math, cd}

\hypersetup{colorlinks=true, allcolors=magenta}
\definecolor{darkGreen}{HTML}{00A000}
\newgeometry{margin = 1in}

\newtheorem{theorem}{Theorem}[section]
\newtheorem{mainTheorem}{Theorem}
\newtheorem{proposition}[theorem]{Proposition}
\newtheorem{lemma}[theorem]{Lemma}
\newtheorem{corollary}[theorem]{Corollary}
\theoremstyle{definition}\newtheorem{example}[theorem]{Example}
\theoremstyle{definition}\newtheorem{definition}[theorem]{Definition}
\theoremstyle{definition}\newtheorem{remark}[theorem]{Remark}
\theoremstyle{definition}\newtheorem{notation}[theorem]{Notation}
\renewcommand{\themainTheorem}{\Alph{mainTheorem}}
\newtheorem*{theorem*}{Theorem}
\newtheorem*{proposition*}{Proposition}
\newtheorem*{lemma*}{Lemma}
\newtheorem*{corollary*}{Corollary}
\theoremstyle{definition}\newtheorem*{example*}{Example}
\theoremstyle{definition}\newtheorem*{definition*}{Definition}
\theoremstyle{definition}\newtheorem*{remark*}{Remark}
\theoremstyle{definition}\newtheorem*{notation*}{Notation}

% Operators
    \newcommand{\id}{\operatorname{id}}
    \newcommand{\im}{\operatorname{im}}
    \newcommand{\rk}{\operatorname{rk}}
    \newcommand{\ch}{\operatorname{ch}}
    \newcommand{\tr}{\operatorname{tr}}
    \newcommand{\tp}{\operatorname{tp}}
    \newcommand{\qd}{\operatorname{qd}}
    \newcommand{\ON}{\operatorname{ON}}
    \newcommand{\GL}{\operatorname{GL}}
    \newcommand{\SL}{\operatorname{SL}}
    \newcommand{\Id}{\operatorname{Id}}
    \newcommand{\Th}{\operatorname{Th}}
    \newcommand{\Cn}{\operatorname{Cn}}
    \newcommand{\Bl}{\operatorname{Bl}}
    \newcommand{\Cl}{\operatorname{Cl}}
    \newcommand{\LT}{\operatorname{LT}}
    \newcommand{\dom}{\operatorname{dom}}
    \newcommand{\ran}{\operatorname{ran}}
    \newcommand{\cdm}{\operatorname{cdm}}
    \newcommand{\sgn}{\operatorname{sgn}}
    \newcommand{\lcm}{\operatorname{lcm}}
    \newcommand{\ord}{\operatorname{ord}}
    \newcommand{\cvx}{\operatorname{cvx}}
    \newcommand{\Aut}{\operatorname{Aut}}
    \newcommand{\Inn}{\operatorname{Inn}}
    \newcommand{\Out}{\operatorname{Out}}
    \newcommand{\End}{\operatorname{End}}
    \newcommand{\Mat}{\operatorname{Mat}}
    \newcommand{\Obj}{\operatorname{Obj}}
    \newcommand{\Hom}{\operatorname{Hom}}
    \newcommand{\Tor}{\operatorname{Tor}}
    \newcommand{\Ann}{\operatorname{Ann}}
    \newcommand{\Sym}{\operatorname{Sym}}
    \newcommand{\Cov}{\operatorname{Cov}}
    \newcommand{\Orb}{\operatorname{Orb}}
    \newcommand{\Sat}{\operatorname{Sat}}
    \newcommand{\Thm}{\operatorname{Thm}}
    \newcommand{\Der}{\operatorname{Der}}
    \newcommand{\Age}{\operatorname{Age}}
    \newcommand{\Div}{\operatorname{Div}}
    \newcommand{\PGL}{\operatorname{PGL}}
    \newcommand{\rank}{\operatorname{rank}}
    \newcommand{\proj}{\operatorname{proj}}
    \newcommand{\diag}{\operatorname{diag}}
    \newcommand{\eval}{\operatorname{eval}}
    \newcommand{\cont}{\operatorname{cont}}
    \newcommand{\diam}{\operatorname{diam}}
    \newcommand{\mult}{\operatorname{mult}}
    \newcommand{\Core}{\operatorname{Core}}
    \newcommand{\Term}{\operatorname{Term}}
    \newcommand{\Taut}{\operatorname{Taut}}
    \newcommand{\Sent}{\operatorname{Sent}}
    \newcommand{\Skew}{\operatorname{Skew}}
    \newcommand{\Frac}{\operatorname{Frac}}
    \newcommand{\Stab}{\operatorname{Stab}}
    \newcommand{\Isom}{\operatorname{Isom}}
    \newcommand{\Meas}{\operatorname{Meas}}
    \newcommand{\Diag}{\operatorname{Diag}}
    \newcommand{\Sing}{\operatorname{Sing}}
    \newcommand{\coker}{\operatorname{coker}}
    \newcommand{\preim}{\operatorname{preim}}
    \newcommand{\Graph}{\operatorname{Graph}}
    \newcommand{\UnSat}{\operatorname{UnSat}}
    \newcommand{\Axioms}{\operatorname{Axioms}}
    \renewcommand{\Re}{\operatorname{Re}}
    \renewcommand{\Im}{\operatorname{Im}}
    \renewcommand{\div}{\operatorname{div}}
    \renewcommand{\span}{\operatorname{span}}
    \renewcommand{\Form}{\operatorname{Form}}

% Math notations
    % Set Theory, Category Theory, and Logic
        \newcommand{\fa}{\forall}
        \newcommand{\ex}{\exists}
        \newcommand{\MP}{\textrm{MP}}
        \newcommand{\PA}{\textrm{PA}}
        \newcommand{\PL}{{\rm P{\small L}}}
        \newcommand{\DLO}{\textrm{DLO}}
        \newcommand{\ZFC}{\textrm{ZFC}}
        \newcommand{\ACF}{\textrm{ACF}}
        \newcommand{\FOL}{{\rm F{\small OL}}}
        \newcommand{\iso}{\cong}
        \newcommand{\pow}{\mathcal{P}}
        \newcommand{\comp}{\setminus}
        \newcommand{\into}{\hookrightarrow}
        \newcommand{\onto}{\twoheadrightarrow}
        \newcommand{\parto}{\rightharpoonup}
        \newcommand{\eqnum}{\approx}
        \newcommand{\natiso}{\simeq}
        \newcommand{\proves}{\vdash}
        \newcommand{\adjoin}{^\smallfrown}
        \newcommand{\nproves}{\nvdash}
        \newcommand{\symdiff}{\vartriangle}
        \newcommand{\infrule}{\rightsquigarrow}
        \newcommand{\eleminto}{\into_e}
        \newcommand{\elemequiv}{\equiv}
        \newcommand{\substruct}{<}
        \newcommand{\supstruct}{>}
        \newcommand{\elemembed}{\preceq}
        \newcommand{\elemextend}{\succeq}
        \newcommand{\substructeq}{\leq}
        \newcommand{\supstructeq}{\geq}
        \renewcommand{\em}{\varnothing}
        \renewcommand{\vec}[1]{\bar{#1}}

    % Complexity Theory
        \newcommand{\NP}{\mathsf{NP}}
        \newcommand{\coNP}{\mathsf{coNP}}

    % Categories
        \newcommand{\cat}[1]{\textbf{#1}}
        \newcommand{\catset}{\cat{Set}}
        \newcommand{\catgrp}{\cat{Grp}}
        \newcommand{\catmon}{\cat{Mon}}
        \newcommand{\cattop}{\cat{Top}}
        \newcommand{\catmet}{\cat{Met}}
        \newcommand{\catrel}{\cat{Rel}}
        \newcommand{\catord}{\cat{Ord}}
        \newcommand{\catscat}{\cat{Cat}}
        \newcommand{\catlscat}{\cat{CAT}}
        \newcommand{\catgrpd}{\cat{Grpd}}
        \newcommand{\catring}{\cat{Ring}}
        \newcommand{\cathtop}{\cat{hTop}}
        \newcommand{\catptop}{\cat{Top}_\blob}
        \newcommand{\catphtop}{\cat{hTop}_\blob}
        \newcommand{\catabgrp}{\cat{Ab}}
        \newcommand{\catman}[1][\infty]{\cat{Man}_{#1}}
        \newcommand{\cathom}[1][\mc{L}]{#1\textrm{-}\cat{Hom}}
        \newcommand{\catemb}[1][\mc{L}]{#1\textrm{-}\cat{Emb}}
        \newcommand{\catmod}[1][R]{\prescript{}{#1}{\cat{Mod}}}
        \newcommand{\catrmod}[1][R]{\cat{Mod}_{#1}}
        \newcommand{\catcov}[1][X]{\cat{Cov}\l(#1\r)}
        \newcommand{\catfgmod}[1][R]{\cat{fg}_{#1}\cat{Mod}}
        \newcommand{\catvect}[1][k]{\prescript{}{#1}{\cat{Vect}}}
        \newcommand{\catfgvect}[1][k]{\cat{fg}_{#1}\cat{Vect}}
        \newcommand{\catalg}[1][R]{\prescript{}{#1}{\cat{Alg}}}
        \newcommand{\catgset}[1]{\prescript{}{#1}{\cat{Set}}}
        \newcommand{\catmodel}[2][\mc{L}]{\mc{M}_#1\!\l(#2\r)}
        \newcommand{\catrep}[2][\,]{\cat{Rep}_{#1\!}\l(#2\r)}
        \newcommand{\catfgrep}[2][\,]{\cat{fgRep}_{#1\!}\l(#2\r)}

    % Analysis
        \newcommand{\BV}{BV}
        \newcommand{\del}{\partial}
        \newcommand{\incto}{\nearrow}
        \newcommand{\decto}{\searrow}
        \newcommand{\abscont}{\ll}
        \newcommand{\esssup}{\operatorname{ess-sup}}
        \renewcommand{\d}{\mathrm{d}}

    % Topology
        \newcommand{\rel}{\,\operatorname{rel}\,}
        \newcommand{\tcl}{\operatorname{cl}}
        \newcommand{\scl}{\operatorname{scl}}
        \newcommand{\tint}{\operatorname{int}}
        \newcommand{\sint}{\operatorname{sint}}
        \newcommand{\htopeq}{\simeq}
        \newcommand{\pathto}{\rightsquigarrow}

    % Linear Algebra
        \newcommand{\dual}{\wedge}
        \newcommand{\adj}{\ast}
        \newcommand{\trans}{\mathsf{T}}
        \newcommand{\inprod}[2]{\l\langle{#1},{#2}\r\rangle}

    % Group Theory
        \newcommand{\act}{\curvearrowright}
        \newcommand{\semi}{\rtimes}
        \newcommand{\nsubgrp}{\triangleleft}
        \newcommand{\nsupgrp}{\triangleright}
        \newcommand{\nsubgrpeq}{\trianglelefteq}
        \newcommand{\nsupgrpeq}{\trianglerighteq}

    % Number Theory
        \newcommand{\divides}{\,|\,}
        \newcommand{\ndivides}{\nmid}
        \renewcommand{\mod}[1]{\l(\operatorname{mod}\,#1\r)}

    % Algebraic Geometry
        \newcommand{\ratto}{\dashrightarrow}

    % Misc
        \newcommand{\st}{:}
        \newcommand{\tpl}[1]{\l(#1\r)}
        \newcommand{\gen}[1]{\l\langle#1\r\rangle}
        \renewcommand{\bar}{\overline}

% Math others
    % Number Systems
        \newcommand{\N}{\mathbb{N}}
        \newcommand{\Z}{\mathbb{Z}}
        \newcommand{\Q}{\mathbb{Q}}
        \newcommand{\R}{\mathbb{R}}
        \newcommand{\C}{\mathbb{C}}
        \newcommand{\F}{\mathbb{F}}
        \newcommand{\E}{\mathbb{E}}
        \newcommand{\A}{\mathbb{A}}
        \renewcommand{\S}{\mathbb{S}}
        \renewcommand{\P}{\mathbb{P}}
        \renewcommand{\H}{\mathbb{H}}

% LaTeX/MathJax
    % Fonts
        \newcommand{\mc}[1]{\mathcal{#1}}
        \newcommand{\ms}[1]{\mathscr{#1}}
        \newcommand{\mb}[1]{\mathbb{#1}}
        \newcommand{\mf}[1]{\mathfrak{#1}}
        \renewcommand{\it}[1]{\textit{#1}}
        \renewcommand{\bf}[1]{\textbf{#1}}
        \renewcommand{\sf}[1]{\textsf{#1}}
        \renewcommand{\phi}{\varphi}
        \renewcommand{\epsilon}{\varepsilon}

    % Meta
        \newcommand{\blob}{\bullet}
        \newcommand{\slot}{-}
        \newcommand{\cref}[1]{\tag{$\,#1\,$}}
        \newcommand{\qedin}{\tag*{$\blacksquare$}}
        \newcommand{\exqedin}{\tag*{$\blacklozenge$}}
        \renewcommand{\l}{\left}
        \renewcommand{\r}{\right}
        \renewcommand{\qed}{\phantom\qedhere\hfill$\blacksquare$}
        \renewcommand{\ref}[1]{\l(\,#1\,\r)}


\begin{document}
    \title{Questions in Lemma 5.19}
    \author{Zhaoshen Zhai}
    \date{\today}
    \maketitle

    \newcommand{\Ball}{\operatorname{Ball}}

    \section{Results from the Paper}

    I'll first reference a previous lemma, and then reproduce Lemma \ref{lem:5.19} and its proof here.

    \setcounter{section}{2}
    \setcounter{theorem}{66}
    \begin{lemma}\label{lem:2.67}
        Let $(X,G)$, $(Y,H)$ be connected locally-finite graphs. For a coarse embedding $f:X\to Y$ and $A\in\mc{H}_{\del<\infty}(Y)$, $\diam(\del_\sf{v}f^{-1}(A))$ is uniformly bounded in terms of $\diam(\del_\sf{v}A)$, and also $f^{-1}(A)\in\mc{H}_{\del<\infty}(X)$.
    \end{lemma}

    \setcounter{section}{5}
    \setcounter{theorem}{18}
    \begin{lemma}\label{lem:5.19}
        The class of connected locally-finite graphs in which $\mc{H}_{\diam(\del)\leq R}$ is dense towards ends for some $R<\infty$ is invariant under coarse equivalence.
    \end{lemma}
    \begin{proof}
        Let $(X,G)$, $(Y,T)$ be connected locally-finite graphs, $f:X\to Y$ be a coarse equivalence with quasi-inverse $g:Y\to X$, and suppose $\mc{H}_{\diam(\del)\leq S}(Y)$ is dense towards ends for some $S<\infty$. By Lemma \ref{lem:2.67}, pick some $R<\infty$ so that for any $H\in\mc{H}_{\diam(\del)\leq S}(Y)$, we have $f^{-1}(H)\in\mc{H}_{\diam(\del)\leq R}(X)$. Then for any $U\in\widehat{X}\comp X$ and $A\in\mc{H}_{\del<\infty}(X)$ with $U\in\widehat{A}$, letting $B\coloneqq\lnot\Ball_{d(1_X,g\circ f)}(\lnot A)$, we have $f^{-1}(g^{-1}(B))\subseteq\Ball_{d(1_X,g\circ f)}(B)\subseteq A$, and $A\symdiff B$, $B\symdiff f^{-1}(g^{-1}(B))$ are finite, so $U\in\widehat{f^{-1}(g^{-1}(B))}$, so $\widehat{f}(U)\in\widehat{g^{-1}(B)}$, so there is $g^{-1}(B)\supseteq H\in\mc{H}_{\diam(\del)\leq S}(Y)$ with {\color{red}{$\widehat{f}(U)\in H$}}\footnote{I think this should be $H\in\widehat{f}(U)$, or equivalently $\widehat{f}(U)\in\widehat{H}$.}, so $f^{-1}(H)\in\mc{H}_{\diam(\del)\leq R}(X)$ with $U\in\widehat{f^{-1}(H)}$ and $f^{-1}(H)\subseteq f^{-1}(g^{-1}(B))\subseteq A$.
    \end{proof}

    \setcounter{section}{1}
    \section{Detailed Proof to Check my Understanding}

    I'll give some details in the proof and rewrite it in a way that I can understand, in order to ask you if my understanding of this proof is correct (as it is a very important step towards a generalization/modification).

    \begin{proof}
        Let $(X,G)$, $(Y,T)$, $f:X\to Y$ and $g:Y\to X$ be as above, $\mc{H}_{\diam(\del)\leq S}(Y)$ be dense towards ends, and $R<\infty$ be so that for any $H\in\mc{H}_{\diam(\del)\leq S}(Y)$, we have $f^{-1}(H)\in\mc{H}_{\diam(\del)\leq R}(X)$.

        Fix an end $U\in\widehat{X}\comp X$ with $U\in\widehat{A}$ for some $A\in\mc{H}_{\del<\infty}(X)$. We need to find some\footnote{Warning: My $B\in\mc{H}_{\del<\infty}(X)$ is \textit{not} the same $B$ as in the original proof.} $B\in\mc{H}_{\del<\infty}(Y)$ such that $\widehat{f}(U)\in\widehat{B}$ and $f^{-1}(B)\subseteq A$, for then $\widehat{f}(U)\in\widehat{H}$ for some $B\supseteq H\in\mc{H}_{\diam(\del)\leq S}(Y)$, and hence we have
        \begin{equation*}
            U\in\widehat{f^{-1}(H)}\subseteq\widehat{f^{-1}(B)}\subseteq\widehat{A}
        \end{equation*}
        with $f^{-1}(H)\in\mc{H}_{\diam(\del)\leq R}(X)$. For convenience, let $D<\infty$ be the uniform distance $d(1_X,g\circ f)$.

        To this end, note that $\widehat{f}(U)\in\widehat{B}$ iff $U\in\widehat{f^{-1}(B)}$. Since $U\in\widehat{A}$, the latter can occur if $|A\symdiff f^{-1}(B)|<\infty$, and so we need to find such a $B\in\mc{H}_{\del<\infty}(Y)$ with the additional property that $f^{-1}(B)\subseteq A$.
        \begin{itemize}
            \item \textit{Attempt 1:} Set $B\coloneqq g^{-1}(A)\in\mc{H}_{\del<\infty}(Y)$. Then $f^{-1}(B)\subseteq\Ball_D(A)$ since if $(g\circ f)(x)\in A$, then
                \begin{equation*}
                    d(x,A)\leq d(x,(g\circ f)(x))\leq d(1_X,g\circ f)=D.
                \end{equation*}
            By local-finiteness of $G$, we see that $A\symdiff f^{-1}(B)=A\comp f^{-1}(B)$ is finite, as desired.
        \end{itemize}
        However, it is \textit{not} the case that $f^{-1}(B)\subseteq A$. To remedy this, we `shrink' $A$ by $D$ to $A'$ so that $\Ball_D(A')\subseteq A$, and take $B\coloneqq g^{-1}(A')$ instead. Indeed, $A'\coloneqq\lnot\Ball_D(\lnot A)\subseteq A$ works, since $f^{-1}(B)\subseteq\Ball_D(A')$ as before, so $A'\symdiff f^{-1}(B)=A'\comp f^{-1}(B)$ is finite. Also, $A\symdiff A'$ is finite since $x\in A\symdiff A'$ iff $x\in A$ and $d(x,\lnot A)\leq D$, so $A\symdiff f^{-1}(B)$ is finite too. It remains to show that $\Ball_D(A')\subseteq A$, for then $f^{-1}(B)\subseteq A$ as desired.

        Indeed, if $y\in\Ball_D(A')$, then by the (reverse) triangle-inequality we have $d(y,\lnot A)\geq d(x,\lnot A)-d(x,y)$ for all $x\in A'$. But $d(x,\lnot A)>D$, strictly, so $d(y,\lnot A)>D-D=0$, and hence $y\in A$.
    \end{proof}
\end{document}
