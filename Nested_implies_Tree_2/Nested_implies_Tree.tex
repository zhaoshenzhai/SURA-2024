\documentclass{amsart}
\usepackage[hidelinks]{hyperref}                                            % links
\usepackage{pgfplots}\pgfplotsset{compat=1.18}                              % plots
\usepackage{amsfonts, amsmath, amssymb, amsthm}                             % basic maths commands
\usepackage{mathtools, mathrsfs}                                            % more maths commands
\usepackage{graphicx}                                                       % images and other graphics
\usepackage{geometry}                                                       % page layout
\usepackage{tikz, tikz-3dplot, tikzpagenodes}                               % maths figures
\usepackage{caption, subcaption}                                            % captions outside float
\usepackage{xcolor}                                                         % more colors
\usepackage{enumitem}                                                       % enumerate and itemize indents
\usepackage{nicematrix}                                                     % matrices and tables

\usetikzlibrary{matrix, positioning, patterns, decorations.markings, arrows, arrows.meta, backgrounds, math, cd}

\hypersetup{colorlinks=true, allcolors=magenta}
\definecolor{darkGreen}{HTML}{00A000}
\newgeometry{margin = 1in}

\newtheorem{theorem}{Theorem}[section]
\newtheorem{mainTheorem}{Theorem}
\newtheorem{proposition}[theorem]{Proposition}
\newtheorem{lemma}[theorem]{Lemma}
\newtheorem{corollary}[theorem]{Corollary}
\theoremstyle{definition}\newtheorem{example}[theorem]{Example}
\theoremstyle{definition}\newtheorem{definition}[theorem]{Definition}
\theoremstyle{definition}\newtheorem{remark}[theorem]{Remark}
\theoremstyle{definition}\newtheorem{notation}[theorem]{Notation}
\renewcommand{\themainTheorem}{\Alph{mainTheorem}}
\newtheorem*{theorem*}{Theorem}
\newtheorem*{proposition*}{Proposition}
\newtheorem*{lemma*}{Lemma}
\newtheorem*{corollary*}{Corollary}
\theoremstyle{definition}\newtheorem*{example*}{Example}
\theoremstyle{definition}\newtheorem*{definition*}{Definition}
\theoremstyle{definition}\newtheorem*{remark*}{Remark}
\theoremstyle{definition}\newtheorem*{notation*}{Notation}

% Operators
    \newcommand{\id}{\operatorname{id}}
    \newcommand{\im}{\operatorname{im}}
    \newcommand{\rk}{\operatorname{rk}}
    \newcommand{\ch}{\operatorname{ch}}
    \newcommand{\tr}{\operatorname{tr}}
    \newcommand{\tp}{\operatorname{tp}}
    \newcommand{\qd}{\operatorname{qd}}
    \newcommand{\ON}{\operatorname{ON}}
    \newcommand{\GL}{\operatorname{GL}}
    \newcommand{\SL}{\operatorname{SL}}
    \newcommand{\Id}{\operatorname{Id}}
    \newcommand{\Th}{\operatorname{Th}}
    \newcommand{\Cn}{\operatorname{Cn}}
    \newcommand{\Bl}{\operatorname{Bl}}
    \newcommand{\Cl}{\operatorname{Cl}}
    \newcommand{\LT}{\operatorname{LT}}
    \newcommand{\dom}{\operatorname{dom}}
    \newcommand{\ran}{\operatorname{ran}}
    \newcommand{\cdm}{\operatorname{cdm}}
    \newcommand{\sgn}{\operatorname{sgn}}
    \newcommand{\lcm}{\operatorname{lcm}}
    \newcommand{\ord}{\operatorname{ord}}
    \newcommand{\cvx}{\operatorname{cvx}}
    \newcommand{\Aut}{\operatorname{Aut}}
    \newcommand{\Inn}{\operatorname{Inn}}
    \newcommand{\Out}{\operatorname{Out}}
    \newcommand{\End}{\operatorname{End}}
    \newcommand{\Mat}{\operatorname{Mat}}
    \newcommand{\Obj}{\operatorname{Obj}}
    \newcommand{\Hom}{\operatorname{Hom}}
    \newcommand{\Tor}{\operatorname{Tor}}
    \newcommand{\Ann}{\operatorname{Ann}}
    \newcommand{\Sym}{\operatorname{Sym}}
    \newcommand{\Cov}{\operatorname{Cov}}
    \newcommand{\Orb}{\operatorname{Orb}}
    \newcommand{\Sat}{\operatorname{Sat}}
    \newcommand{\Thm}{\operatorname{Thm}}
    \newcommand{\Der}{\operatorname{Der}}
    \newcommand{\Age}{\operatorname{Age}}
    \newcommand{\Div}{\operatorname{Div}}
    \newcommand{\PGL}{\operatorname{PGL}}
    \newcommand{\rank}{\operatorname{rank}}
    \newcommand{\proj}{\operatorname{proj}}
    \newcommand{\diag}{\operatorname{diag}}
    \newcommand{\eval}{\operatorname{eval}}
    \newcommand{\cont}{\operatorname{cont}}
    \newcommand{\diam}{\operatorname{diam}}
    \newcommand{\mult}{\operatorname{mult}}
    \newcommand{\Core}{\operatorname{Core}}
    \newcommand{\Term}{\operatorname{Term}}
    \newcommand{\Taut}{\operatorname{Taut}}
    \newcommand{\Sent}{\operatorname{Sent}}
    \newcommand{\Skew}{\operatorname{Skew}}
    \newcommand{\Frac}{\operatorname{Frac}}
    \newcommand{\Stab}{\operatorname{Stab}}
    \newcommand{\Isom}{\operatorname{Isom}}
    \newcommand{\Meas}{\operatorname{Meas}}
    \newcommand{\Diag}{\operatorname{Diag}}
    \newcommand{\Sing}{\operatorname{Sing}}
    \newcommand{\coker}{\operatorname{coker}}
    \newcommand{\preim}{\operatorname{preim}}
    \newcommand{\Graph}{\operatorname{Graph}}
    \newcommand{\UnSat}{\operatorname{UnSat}}
    \newcommand{\Axioms}{\operatorname{Axioms}}
    \renewcommand{\Re}{\operatorname{Re}}
    \renewcommand{\Im}{\operatorname{Im}}
    \renewcommand{\div}{\operatorname{div}}
    \renewcommand{\span}{\operatorname{span}}
    \renewcommand{\Form}{\operatorname{Form}}

% Math notations
    % Set Theory, Category Theory, and Logic
        \newcommand{\fa}{\forall}
        \newcommand{\ex}{\exists}
        \newcommand{\MP}{\textrm{MP}}
        \newcommand{\PA}{\textrm{PA}}
        \newcommand{\PL}{{\rm P{\small L}}}
        \newcommand{\DLO}{\textrm{DLO}}
        \newcommand{\ZFC}{\textrm{ZFC}}
        \newcommand{\ACF}{\textrm{ACF}}
        \newcommand{\FOL}{{\rm F{\small OL}}}
        \newcommand{\iso}{\cong}
        \newcommand{\pow}{\mathcal{P}}
        \newcommand{\comp}{\setminus}
        \newcommand{\into}{\hookrightarrow}
        \newcommand{\onto}{\twoheadrightarrow}
        \newcommand{\parto}{\rightharpoonup}
        \newcommand{\eqnum}{\approx}
        \newcommand{\natiso}{\simeq}
        \newcommand{\proves}{\vdash}
        \newcommand{\adjoin}{^\smallfrown}
        \newcommand{\nproves}{\nvdash}
        \newcommand{\symdiff}{\vartriangle}
        \newcommand{\infrule}{\rightsquigarrow}
        \newcommand{\eleminto}{\into_e}
        \newcommand{\elemequiv}{\equiv}
        \newcommand{\substruct}{<}
        \newcommand{\supstruct}{>}
        \newcommand{\elemembed}{\preceq}
        \newcommand{\elemextend}{\succeq}
        \newcommand{\substructeq}{\leq}
        \newcommand{\supstructeq}{\geq}
        \renewcommand{\em}{\varnothing}
        \renewcommand{\vec}[1]{\bar{#1}}

    % Complexity Theory
        \newcommand{\NP}{\mathsf{NP}}
        \newcommand{\coNP}{\mathsf{coNP}}

    % Categories
        \newcommand{\cat}[1]{\textbf{#1}}
        \newcommand{\catset}{\cat{Set}}
        \newcommand{\catgrp}{\cat{Grp}}
        \newcommand{\catmon}{\cat{Mon}}
        \newcommand{\cattop}{\cat{Top}}
        \newcommand{\catmet}{\cat{Met}}
        \newcommand{\catrel}{\cat{Rel}}
        \newcommand{\catord}{\cat{Ord}}
        \newcommand{\catscat}{\cat{Cat}}
        \newcommand{\catlscat}{\cat{CAT}}
        \newcommand{\catgrpd}{\cat{Grpd}}
        \newcommand{\catring}{\cat{Ring}}
        \newcommand{\cathtop}{\cat{hTop}}
        \newcommand{\catptop}{\cat{Top}_\blob}
        \newcommand{\catphtop}{\cat{hTop}_\blob}
        \newcommand{\catabgrp}{\cat{Ab}}
        \newcommand{\catman}[1][\infty]{\cat{Man}_{#1}}
        \newcommand{\cathom}[1][\mc{L}]{#1\textrm{-}\cat{Hom}}
        \newcommand{\catemb}[1][\mc{L}]{#1\textrm{-}\cat{Emb}}
        \newcommand{\catmod}[1][R]{\prescript{}{#1}{\cat{Mod}}}
        \newcommand{\catrmod}[1][R]{\cat{Mod}_{#1}}
        \newcommand{\catcov}[1][X]{\cat{Cov}\l(#1\r)}
        \newcommand{\catfgmod}[1][R]{\cat{fg}_{#1}\cat{Mod}}
        \newcommand{\catvect}[1][k]{\prescript{}{#1}{\cat{Vect}}}
        \newcommand{\catfgvect}[1][k]{\cat{fg}_{#1}\cat{Vect}}
        \newcommand{\catalg}[1][R]{\prescript{}{#1}{\cat{Alg}}}
        \newcommand{\catgset}[1]{\prescript{}{#1}{\cat{Set}}}
        \newcommand{\catmodel}[2][\mc{L}]{\mc{M}_#1\!\l(#2\r)}
        \newcommand{\catrep}[2][\,]{\cat{Rep}_{#1\!}\l(#2\r)}
        \newcommand{\catfgrep}[2][\,]{\cat{fgRep}_{#1\!}\l(#2\r)}

    % Analysis
        \newcommand{\BV}{BV}
        \newcommand{\del}{\partial}
        \newcommand{\incto}{\nearrow}
        \newcommand{\decto}{\searrow}
        \newcommand{\abscont}{\ll}
        \newcommand{\esssup}{\operatorname{ess-sup}}
        \renewcommand{\d}{\mathrm{d}}

    % Topology
        \newcommand{\rel}{\,\operatorname{rel}\,}
        \newcommand{\tcl}{\operatorname{cl}}
        \newcommand{\scl}{\operatorname{scl}}
        \newcommand{\tint}{\operatorname{int}}
        \newcommand{\sint}{\operatorname{sint}}
        \newcommand{\htopeq}{\simeq}
        \newcommand{\pathto}{\rightsquigarrow}

    % Linear Algebra
        \newcommand{\dual}{\wedge}
        \newcommand{\adj}{\ast}
        \newcommand{\trans}{\mathsf{T}}
        \newcommand{\inprod}[2]{\l\langle{#1},{#2}\r\rangle}

    % Group Theory
        \newcommand{\act}{\curvearrowright}
        \newcommand{\semi}{\rtimes}
        \newcommand{\nsubgrp}{\triangleleft}
        \newcommand{\nsupgrp}{\triangleright}
        \newcommand{\nsubgrpeq}{\trianglelefteq}
        \newcommand{\nsupgrpeq}{\trianglerighteq}

    % Number Theory
        \newcommand{\divides}{\,|\,}
        \newcommand{\ndivides}{\nmid}
        \renewcommand{\mod}[1]{\l(\operatorname{mod}\,#1\r)}

    % Algebraic Geometry
        \newcommand{\ratto}{\dashrightarrow}

    % Misc
        \newcommand{\st}{:}
        \newcommand{\tpl}[1]{\l(#1\r)}
        \newcommand{\gen}[1]{\l\langle#1\r\rangle}
        \renewcommand{\bar}{\overline}

% Math others
    % Number Systems
        \newcommand{\N}{\mathbb{N}}
        \newcommand{\Z}{\mathbb{Z}}
        \newcommand{\Q}{\mathbb{Q}}
        \newcommand{\R}{\mathbb{R}}
        \newcommand{\C}{\mathbb{C}}
        \newcommand{\F}{\mathbb{F}}
        \newcommand{\E}{\mathbb{E}}
        \newcommand{\A}{\mathbb{A}}
        \renewcommand{\S}{\mathbb{S}}
        \renewcommand{\P}{\mathbb{P}}
        \renewcommand{\H}{\mathbb{H}}

% LaTeX/MathJax
    % Fonts
        \newcommand{\mc}[1]{\mathcal{#1}}
        \newcommand{\ms}[1]{\mathscr{#1}}
        \newcommand{\mb}[1]{\mathbb{#1}}
        \newcommand{\mf}[1]{\mathfrak{#1}}
        \renewcommand{\it}[1]{\textit{#1}}
        \renewcommand{\bf}[1]{\textbf{#1}}
        \renewcommand{\sf}[1]{\textsf{#1}}
        \renewcommand{\phi}{\varphi}
        \renewcommand{\epsilon}{\varepsilon}

    % Meta
        \newcommand{\blob}{\bullet}
        \newcommand{\slot}{-}
        \newcommand{\cref}[1]{\tag{$\,#1\,$}}
        \newcommand{\qedin}{\tag*{$\blacksquare$}}
        \newcommand{\exqedin}{\tag*{$\blacklozenge$}}
        \renewcommand{\l}{\left}
        \renewcommand{\r}{\right}
        \renewcommand{\qed}{\phantom\qedhere\hfill$\blacksquare$}
        \renewcommand{\ref}[1]{\l(\,#1\,\r)}


\begin{document}
    \title{Tree of Orientations on a Nested Collection of Sets}
    \author{Zhaoshen Zhai}
    \date{\today}
    \maketitle

    Let $H\subseteq2^X$ be a sub-pocset for some fixed set $X$ (so that, in particular, $H$ is closed under complements). With the definitions in Section \ref{prelim}, we prove the following
    \begin{mainTheorem}[Propositions \ref{acyclic}, \ref{tree-fs}, \ref{tree-cv}]\label{main}
        If $H$ is nested, then the graph $\mc{T}_H$, whose:
        \begin{itemize}
            \item Vertices are finitely-based orientations $p\subseteq H$;
            \item Edges are pairs $\l\{p,q\r\}$ such that $q=p\symdiff\l\{h,\lnot h\r\}$ for some $\subseteq$-minimal $h\in H$;
        \end{itemize}
        is acyclic. Furthermore, if $H$ is finitely-separating (or more generally, chain-vanishing), then $\mc{T}_H$ is a tree.
    \end{mainTheorem}

    In particular, this applies to when $(X,G)$ is a graph and $H$ is a nested collection of cuts on $X$. No further assumptions on $X$ (like local-finiteness) is needed.

    \section{Preliminaries}\label{prelim}

    Let $H\subseteq2^X$ be a sub-pocset for some fixed set $X$, whose elements $h\in H$ we call \textit{half-spaces}.

    \begin{definition}
        Two elements $h,k\in H$ are \textit{nested} if $h^i\cap k^j=\em$ for some $i,j\in\l\{1,-1\r\}$, where $h^i\coloneqq h$ for $i=1$ and $h^i\coloneqq h^c$ otherwise. We say that $H$ is \textit{nested} if every pair $h,k\in H$ are nested.
    \end{definition}

    \subsection{Orientations}

    We give the standard definition of orientations on $H$ and characterize them as `consistent assignments of half-spaces to hyperplanes'.

    \begin{definition}
        An \textit{orientation} on $H$ is a subset $U\subseteq H$ such that
        \begin{enumerate}
            \item[1.] \textit{(Upward-closure)}. If $h\in U$ and $k\in H$ contains $h$, then $k\in U$.
            \item[2.] \textit{(Ultra)}. For each $h\in H$, exactly one of $h,h^c$ is contained in $U$.
        \end{enumerate}
    \end{definition}

    Consider the equivalence relation $\sim$ on $H$ generated by $h\sim h^c$ for all $h\in H$, whose classes are called \textit{hyperplanes} $\del h\coloneqq\l\{h,h^c\r\}$ where $\del:H\to H/\!\sim$ is the projection. We show that an orientation on $H$ is just a choice $\phi:H/\!\sim\,\to H$ of a half-space for each hyperplane, that is consistent in the sense below.

    \begin{proposition}\label{correspondence}
        An orientation $p\subseteq H$ is exactly the data of a function $\phi:H/\!\sim\,\to H$ such that $\phi(\del h)\in\del h$ and $\phi(\del h)\not\subseteq\phi(\del k)^c$ for every $h,k\in H$.
    \end{proposition}
    \begin{proof}
        Given an orientation $p\subseteq H$, let $\phi_p(\del h)\coloneqq h^i\in U$ for the unique $i\in\l\{1,-1\r\}$. That $\phi_p(\del h)\in\del h$ is clear, and if $\phi_p(\del h)\subseteq\phi_p(\del k)^c$, then $U$ contains both $\phi_p(\del k)$ and $\phi_p(\del k)^c$ by upward-closure, a contradiction.

        Conversely, given such a function $\phi:H/\!\sim\,\to H$, let $p_\phi\coloneqq\im\phi\subseteq H$. This is ultra since if $h\in H$ and $h^c\not\in p_\phi$, then $\phi(\del h)\in\del h=\l\{h,h^c\r\}$ implies $h\in p_\phi$. Furthermore, if $p_\phi\ni h\subseteq k$, then $k^c\in p_\phi$ implies $\phi(\del h)=h\subseteq k=\phi(\del k)^c$, a contradiction, so $k\in p_\phi$ by the above.

        Finally, given an orientation $p\subseteq H$, we have $h\in p$ iff $\phi_p(\del h)=h$, which occurs iff $h\in\im\phi_p=p_{\phi_p}$. Thus $p_{\phi_p}=p$. Conversely, given $\phi:H/\!\sim\,\to H$, and $h\in H$, we have $\phi_{p_{\phi}}(\del h)=h^i$ iff $h^i\in p_\phi=\im\phi$, which occurs iff $\phi(\del h)=h^i$. Thus $\phi_{p_\phi}=\phi$ too, as desired.
    \end{proof}

    \begin{definition}
        A \textit{base} for an orientation $p\subseteq H$ is a $\subseteq$-minimal subset $p_0\subseteq p$ such that $p=\,\,\uparrow\!p_0$, where
        \begin{equation*}
            \uparrow\! p_0\coloneqq\bigcup_{h\in p_0}\uparrow\! h\coloneqq\bigcup_{h\in p_0}\l\{k\in H\st k\supseteq h\r\}.
        \end{equation*}
        We say that $p$ is \textit{finitely-based} if it admits a finite basis.
    \end{definition}

    In the above correspondence, a base for $\phi:H/\!\sim\,\to H$ is a function $\phi_0\subseteq\phi$ where $\dom\phi_0$ is a $\subseteq$-minimal subset of hyperplanes such that $\phi_0$ extends uniquely to $\phi$. Thus, the finitely-based orientations are the ones determined by a choice of half-spaces from finitely-many hyperplanes.

    \subsection{Flipping basis elements}

    We now investigate the behaviour of orientation when its choice on a single half-space is modified. Although the proofs work without the characterization in Proposition \ref{correspondence}, it makes orientations a lot more intuitive to me, and so will be phrased this way.

    \begin{lemma}\label{tree-well-defined}
        Let $p\subseteq H$ be an orientation. Then $q\coloneqq p\symdiff\l\{h,h^c\r\}$ is an orientation iff $h\in p$ is $\subseteq$-minimal.

        Furthermore, if $p$ is finitely-based, then so is $q$.
    \end{lemma}
    \begin{proof}
        First, note that $\phi_q(\del h)=\phi_p(h)^c=h^c$ and $\phi_q=\phi_p$ away from $\del h$.
        \begin{itemize}
            \item[($\Rightarrow$).] If $q$ is an orientation and $p\ni k\subset h$, then we have a contradiction since
                \begin{equation*}
                    \phi_q(\del k)=\phi_p(\del k)=k\subseteq h=\phi_q(\del h)^c.
                \end{equation*}
            \item[($\Leftarrow$).]Conversely, let $h\in p$ be $\subseteq$-minimal and suppose that $q$ is not an orientation. Since only $\phi_q(\del h)=h^c$ differs from $\phi_p$, this can only occur if $\phi_q(\del h)\subseteq\phi_q(\del k)^c$ for some $k\in H$. But then
                \begin{equation*}
                    h^c=\phi_q(\del h)\subseteq\phi_q(\del k)^c=k^i\not\in p
                \end{equation*}
                for some unique $i\in\l\{1,-1\r\}$, so that $p\ni k^{-i}\subseteq h$ and contradicts that $h\in p$ is $\subseteq$-minimal.
        \end{itemize}
        Finally, suppose that $p$ is finitely-based. {\color{red}{???}}
    \end{proof}

    \begin{remark}
        If $h\in p$ is $\subseteq$-minimal and $p_0\subseteq p$ is any basis, then $h\in p_0$. Indeed, if $h\not\in p_0$, then there is some $h_0\in p_0\subseteq p$ with $h\supset h_0$, contradicting minimality.
    \end{remark}

    \section{The graph $\mc{T}_H$}

    Let $H\subseteq2^X$ be a nested sub-pocset as before. Using Lemma \ref{tree-well-defined}, we construct a graph $\mc{T}_H$ whose:
    \begin{itemize}
        \item \textit{Vertices} are finitely-based orientations $p\subseteq H$;
        \item \textit{Edges} are pairs $\l\{p,q\r\}$ such that $q=p\symdiff\l\{h,h^c\r\}$ for some $\subseteq$-minimal $h\in H$.
    \end{itemize}

    The goal of this section is to establish Theorem \ref{main}, stating that $\mc{T}_H$ is acyclic (Proposition \ref{acyclic}), and furthermore, that $\mc{T}_H$ is a tree when $P$ is finitely-separating (Proposition \ref{tree-fs}), or more generally, when $P$ is chain-vanishing (Proposition \ref{tree-cv}).

    \subsection{Acyclicity of $\mc{T}_H$}

    Essentially, this is because any path in $\mc{T}_H$ (without backtracking) is induced by a sequence of the flipped half-spaces, and those form a \textit{strictly}-increasing chain.

    \begin{definition}
        Fix $p_0\in V(\mc{T}_H)$ and $n\in\N$. A sequence $(h_i)_{i<n}\subseteq H$ is said to \textit{induce a path from $p_0$} if $(p_i)_{i<n}$, defined by $p_i\coloneqq p_{i-1}\symdiff\{h_{i-1},h_{i-1}^c\}$ for every $1\leq i<n$, is a path in $\mc{T}_H$ with each $h_i\in p_i$.
    \end{definition}

    \begin{remark}
        Any path in $\mc{T}_H$ is induced by its sequence of flipped basis elements.
    \end{remark}

    \begin{lemma}\label{no-backtrack}
        A path from $p_0$ induced by $(h_i)_{i<n}$, $n\geq3$, has no backtracking iff $h_i\neq h_{i-1}^c$ for every $1\leq i<n$.
    \end{lemma}
    \begin{proof}
        Take $2\leq i\leq n$. It suffices to show that $p_{i-2}=p_i$ iff $h_{i-1}=h_{i-2}^c$.
        \begin{itemize}
            \item[($\Rightarrow$).] We have by definition that $p_i=p_{i-2}\cup\{h_{i-1}^c,h_{i-2}^c\}\comp\l\{h_{i-1},h_{i-2}\r\}$, so since $h_{i-2}\in p_{i-2}=p_i$, we have $h_{i-2}=h_{i-1}^c$ as desired.
            \item[($\Leftarrow$).] Again by definition, by noting that the basis-flipping cancels out.\qed
        \end{itemize}
    \end{proof}

    \begin{lemma}\label{strictly-inc}
        If $(h_i)_{i<n}$ induces a path in $\mc{T}_H$ with no backtracking, then $(h_i)_{i<n}$ is strictly increasing.
    \end{lemma}
    \begin{proof}
        By Lemma \ref{no-backtrack}, we have $h_i\neq h_{i-1}^c$ for every $1\leq i<n$. Thus, since $h_i\in p_i=p_{i-1}\cup\{h_{i-1}^c\}\comp\l\{h_{i-1}\r\}$, we see that $h_i\in p_{i-1}$. Clearly $h_i\neq h_{i-1}$. It suffices to remove the three cases when $h_i\subseteq h_{i-1}$, $h_{i-1}\subseteq h_i^c$, and $h_i^c\subseteq h_{i-1}$, since then nestedness of $H$ gives us $h_{i-1}\subsetneq h_i$, as desired.
        \begin{itemize}
            \item If $h_i\subseteq h_{i-1}$, then $h_{i-1}\in p_i$, contradicting the definition of $p_i$.
            \item If $h_{i-1}\subseteq h_i^c$, then $h_i^c\in p_{i-1}$ by upward-closure of $p_{i-1}$, a contradiction.
            \item If $h_i^c\subseteq h_{i-1}$, then $h_{i-1}\in p_{i+1}$ by upward-closure of $p_{i+1}\ni h_i^c$. But since $h_{i-1}\neq h_i^c$, we have by definition of $p_{i+1}$ that $h_{i-1}\in p_i$, a contradiction.\qed
        \end{itemize}
    \end{proof}

    \begin{proposition}\label{acyclic}
        $\mc{T}_H$ is acyclic.
    \end{proposition}
    \begin{proof}
        Let $(p_i)_{i<n}$ be a cycle in $\mc{T}_H$ induced by $(h_i)_{i<n}$. Since cycles are non-backtracking, we have $h_0\subsetneq h_0$ by Lemma \ref{strictly-inc}, a contradiction.
    \end{proof}

    \begin{proposition}\label{tree-fs}
        If $P$ is chain-vanishing, then $\mc{T}_H$ is connected (and hence a tree).
    \end{proposition}

    \begin{proposition}\label{tree-cv}
        
    \end{proposition}

    {\color{red}{
        \begin{proposition}%\label{tree}
            If $P$ is closed under complements, then $\mc{T}_H$ is connected (and hence a tree).
        \end{proposition}
        \begin{proof}
            If $p,p'\in V(\mc{T}_H)$ are two finitely-based orientations on $P$, then swapping their basis elements one by one as in Remark ? gives us a path between $p$ and $p'$; the closure of $P$ is needed to ensure that those basis elements induce a path between the vertices, in that their complement lies in $P$.
        \end{proof}
    }}
\end{document}
