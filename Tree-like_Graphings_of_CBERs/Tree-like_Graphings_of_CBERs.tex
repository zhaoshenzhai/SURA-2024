\documentclass[reqno]{amsart}
\usepackage[hidelinks]{hyperref}                                            % links
\usepackage{pgfplots}\pgfplotsset{compat=1.18}                              % plots
\usepackage{amsfonts, amsmath, amssymb, amsthm}                             % basic maths commands
\usepackage{mathtools, mathrsfs}                                            % more maths commands
\usepackage{graphicx}                                                       % images and other graphics
\usepackage{geometry}                                                       % page layout
\usepackage{tikz, tikz-3dplot, tikzpagenodes}                               % maths figures
\usepackage{caption, subcaption}                                            % captions outside float
\usepackage{xcolor}                                                         % more colors
\usepackage{enumitem}                                                       % enumerate and itemize indents
\usepackage{nicematrix}                                                     % matrices and tables

\usetikzlibrary{matrix, positioning, patterns, decorations.markings, arrows, arrows.meta, backgrounds, math, cd}

\hypersetup{colorlinks=true, allcolors=magenta}
\definecolor{darkGreen}{HTML}{00A000}
\newgeometry{margin = 1in}

\newtheorem{theorem}{Theorem}[section]
\newtheorem{mainTheorem}{Theorem}
\newtheorem{proposition}[theorem]{Proposition}
\newtheorem{lemma}[theorem]{Lemma}
\newtheorem{corollary}[theorem]{Corollary}
\theoremstyle{definition}\newtheorem{example}[theorem]{Example}
\theoremstyle{definition}\newtheorem{definition}[theorem]{Definition}
\theoremstyle{definition}\newtheorem{remark}[theorem]{Remark}
\theoremstyle{definition}\newtheorem{notation}[theorem]{Notation}
\renewcommand{\themainTheorem}{\Alph{mainTheorem}}
\newtheorem*{theorem*}{Theorem}
\newtheorem*{proposition*}{Proposition}
\newtheorem*{lemma*}{Lemma}
\newtheorem*{corollary*}{Corollary}
\theoremstyle{definition}\newtheorem*{example*}{Example}
\theoremstyle{definition}\newtheorem*{definition*}{Definition}
\theoremstyle{definition}\newtheorem*{remark*}{Remark}
\theoremstyle{definition}\newtheorem*{notation*}{Notation}

% Operators
    \newcommand{\id}{\operatorname{id}}
    \newcommand{\im}{\operatorname{im}}
    \newcommand{\rk}{\operatorname{rk}}
    \newcommand{\ch}{\operatorname{ch}}
    \newcommand{\tr}{\operatorname{tr}}
    \newcommand{\tp}{\operatorname{tp}}
    \newcommand{\qd}{\operatorname{qd}}
    \newcommand{\ON}{\operatorname{ON}}
    \newcommand{\GL}{\operatorname{GL}}
    \newcommand{\SL}{\operatorname{SL}}
    \newcommand{\Id}{\operatorname{Id}}
    \newcommand{\Th}{\operatorname{Th}}
    \newcommand{\Cn}{\operatorname{Cn}}
    \newcommand{\Bl}{\operatorname{Bl}}
    \newcommand{\Cl}{\operatorname{Cl}}
    \newcommand{\LT}{\operatorname{LT}}
    \newcommand{\dom}{\operatorname{dom}}
    \newcommand{\ran}{\operatorname{ran}}
    \newcommand{\cdm}{\operatorname{cdm}}
    \newcommand{\sgn}{\operatorname{sgn}}
    \newcommand{\lcm}{\operatorname{lcm}}
    \newcommand{\ord}{\operatorname{ord}}
    \newcommand{\cvx}{\operatorname{cvx}}
    \newcommand{\Aut}{\operatorname{Aut}}
    \newcommand{\Inn}{\operatorname{Inn}}
    \newcommand{\Out}{\operatorname{Out}}
    \newcommand{\End}{\operatorname{End}}
    \newcommand{\Mat}{\operatorname{Mat}}
    \newcommand{\Obj}{\operatorname{Obj}}
    \newcommand{\Hom}{\operatorname{Hom}}
    \newcommand{\Tor}{\operatorname{Tor}}
    \newcommand{\Ann}{\operatorname{Ann}}
    \newcommand{\Sym}{\operatorname{Sym}}
    \newcommand{\Cov}{\operatorname{Cov}}
    \newcommand{\Orb}{\operatorname{Orb}}
    \newcommand{\Sat}{\operatorname{Sat}}
    \newcommand{\Thm}{\operatorname{Thm}}
    \newcommand{\Der}{\operatorname{Der}}
    \newcommand{\Age}{\operatorname{Age}}
    \newcommand{\Div}{\operatorname{Div}}
    \newcommand{\PGL}{\operatorname{PGL}}
    \newcommand{\rank}{\operatorname{rank}}
    \newcommand{\proj}{\operatorname{proj}}
    \newcommand{\diag}{\operatorname{diag}}
    \newcommand{\eval}{\operatorname{eval}}
    \newcommand{\cont}{\operatorname{cont}}
    \newcommand{\diam}{\operatorname{diam}}
    \newcommand{\mult}{\operatorname{mult}}
    \newcommand{\Core}{\operatorname{Core}}
    \newcommand{\Term}{\operatorname{Term}}
    \newcommand{\Taut}{\operatorname{Taut}}
    \newcommand{\Sent}{\operatorname{Sent}}
    \newcommand{\Skew}{\operatorname{Skew}}
    \newcommand{\Frac}{\operatorname{Frac}}
    \newcommand{\Stab}{\operatorname{Stab}}
    \newcommand{\Isom}{\operatorname{Isom}}
    \newcommand{\Meas}{\operatorname{Meas}}
    \newcommand{\Diag}{\operatorname{Diag}}
    \newcommand{\Sing}{\operatorname{Sing}}
    \newcommand{\coker}{\operatorname{coker}}
    \newcommand{\preim}{\operatorname{preim}}
    \newcommand{\Graph}{\operatorname{Graph}}
    \newcommand{\UnSat}{\operatorname{UnSat}}
    \newcommand{\Axioms}{\operatorname{Axioms}}
    \renewcommand{\Re}{\operatorname{Re}}
    \renewcommand{\Im}{\operatorname{Im}}
    \renewcommand{\div}{\operatorname{div}}
    \renewcommand{\span}{\operatorname{span}}
    \renewcommand{\Form}{\operatorname{Form}}

% Math notations
    % Set Theory, Category Theory, and Logic
        \newcommand{\fa}{\forall}
        \newcommand{\ex}{\exists}
        \newcommand{\MP}{\textrm{MP}}
        \newcommand{\PA}{\textrm{PA}}
        \newcommand{\PL}{{\rm P{\small L}}}
        \newcommand{\DLO}{\textrm{DLO}}
        \newcommand{\ZFC}{\textrm{ZFC}}
        \newcommand{\ACF}{\textrm{ACF}}
        \newcommand{\FOL}{{\rm F{\small OL}}}
        \newcommand{\iso}{\cong}
        \newcommand{\pow}{\mathcal{P}}
        \newcommand{\comp}{\setminus}
        \newcommand{\into}{\hookrightarrow}
        \newcommand{\onto}{\twoheadrightarrow}
        \newcommand{\parto}{\rightharpoonup}
        \newcommand{\eqnum}{\approx}
        \newcommand{\natiso}{\simeq}
        \newcommand{\proves}{\vdash}
        \newcommand{\adjoin}{^\smallfrown}
        \newcommand{\nproves}{\nvdash}
        \newcommand{\symdiff}{\vartriangle}
        \newcommand{\infrule}{\rightsquigarrow}
        \newcommand{\eleminto}{\into_e}
        \newcommand{\elemequiv}{\equiv}
        \newcommand{\substruct}{<}
        \newcommand{\supstruct}{>}
        \newcommand{\elemembed}{\preceq}
        \newcommand{\elemextend}{\succeq}
        \newcommand{\substructeq}{\leq}
        \newcommand{\supstructeq}{\geq}
        \renewcommand{\em}{\varnothing}
        \renewcommand{\vec}[1]{\bar{#1}}

    % Complexity Theory
        \newcommand{\NP}{\mathsf{NP}}
        \newcommand{\coNP}{\mathsf{coNP}}

    % Categories
        \newcommand{\cat}[1]{\textbf{#1}}
        \newcommand{\catset}{\cat{Set}}
        \newcommand{\catgrp}{\cat{Grp}}
        \newcommand{\catmon}{\cat{Mon}}
        \newcommand{\cattop}{\cat{Top}}
        \newcommand{\catmet}{\cat{Met}}
        \newcommand{\catrel}{\cat{Rel}}
        \newcommand{\catord}{\cat{Ord}}
        \newcommand{\catscat}{\cat{Cat}}
        \newcommand{\catlscat}{\cat{CAT}}
        \newcommand{\catgrpd}{\cat{Grpd}}
        \newcommand{\catring}{\cat{Ring}}
        \newcommand{\cathtop}{\cat{hTop}}
        \newcommand{\catptop}{\cat{Top}_\blob}
        \newcommand{\catphtop}{\cat{hTop}_\blob}
        \newcommand{\catabgrp}{\cat{Ab}}
        \newcommand{\catman}[1][\infty]{\cat{Man}_{#1}}
        \newcommand{\cathom}[1][\mc{L}]{#1\textrm{-}\cat{Hom}}
        \newcommand{\catemb}[1][\mc{L}]{#1\textrm{-}\cat{Emb}}
        \newcommand{\catmod}[1][R]{\prescript{}{#1}{\cat{Mod}}}
        \newcommand{\catrmod}[1][R]{\cat{Mod}_{#1}}
        \newcommand{\catcov}[1][X]{\cat{Cov}\l(#1\r)}
        \newcommand{\catfgmod}[1][R]{\cat{fg}_{#1}\cat{Mod}}
        \newcommand{\catvect}[1][k]{\prescript{}{#1}{\cat{Vect}}}
        \newcommand{\catfgvect}[1][k]{\cat{fg}_{#1}\cat{Vect}}
        \newcommand{\catalg}[1][R]{\prescript{}{#1}{\cat{Alg}}}
        \newcommand{\catgset}[1]{\prescript{}{#1}{\cat{Set}}}
        \newcommand{\catmodel}[2][\mc{L}]{\mc{M}_#1\!\l(#2\r)}
        \newcommand{\catrep}[2][\,]{\cat{Rep}_{#1\!}\l(#2\r)}
        \newcommand{\catfgrep}[2][\,]{\cat{fgRep}_{#1\!}\l(#2\r)}

    % Analysis
        \newcommand{\BV}{BV}
        \newcommand{\del}{\partial}
        \newcommand{\incto}{\nearrow}
        \newcommand{\decto}{\searrow}
        \newcommand{\abscont}{\ll}
        \newcommand{\esssup}{\operatorname{ess-sup}}
        \renewcommand{\d}{\mathrm{d}}

    % Topology
        \newcommand{\rel}{\,\operatorname{rel}\,}
        \newcommand{\tcl}{\operatorname{cl}}
        \newcommand{\scl}{\operatorname{scl}}
        \newcommand{\tint}{\operatorname{int}}
        \newcommand{\sint}{\operatorname{sint}}
        \newcommand{\htopeq}{\simeq}
        \newcommand{\pathto}{\rightsquigarrow}

    % Linear Algebra
        \newcommand{\dual}{\wedge}
        \newcommand{\adj}{\ast}
        \newcommand{\trans}{\mathsf{T}}
        \newcommand{\inprod}[2]{\l\langle{#1},{#2}\r\rangle}

    % Group Theory
        \newcommand{\act}{\curvearrowright}
        \newcommand{\semi}{\rtimes}
        \newcommand{\nsubgrp}{\triangleleft}
        \newcommand{\nsupgrp}{\triangleright}
        \newcommand{\nsubgrpeq}{\trianglelefteq}
        \newcommand{\nsupgrpeq}{\trianglerighteq}

    % Number Theory
        \newcommand{\divides}{\,|\,}
        \newcommand{\ndivides}{\nmid}
        \renewcommand{\mod}[1]{\l(\operatorname{mod}\,#1\r)}

    % Algebraic Geometry
        \newcommand{\ratto}{\dashrightarrow}

    % Misc
        \newcommand{\st}{:}
        \newcommand{\tpl}[1]{\l(#1\r)}
        \newcommand{\gen}[1]{\l\langle#1\r\rangle}
        \renewcommand{\bar}{\overline}

% Math others
    % Number Systems
        \newcommand{\N}{\mathbb{N}}
        \newcommand{\Z}{\mathbb{Z}}
        \newcommand{\Q}{\mathbb{Q}}
        \newcommand{\R}{\mathbb{R}}
        \newcommand{\C}{\mathbb{C}}
        \newcommand{\F}{\mathbb{F}}
        \newcommand{\E}{\mathbb{E}}
        \newcommand{\A}{\mathbb{A}}
        \renewcommand{\S}{\mathbb{S}}
        \renewcommand{\P}{\mathbb{P}}
        \renewcommand{\H}{\mathbb{H}}

% LaTeX/MathJax
    % Fonts
        \newcommand{\mc}[1]{\mathcal{#1}}
        \newcommand{\ms}[1]{\mathscr{#1}}
        \newcommand{\mb}[1]{\mathbb{#1}}
        \newcommand{\mf}[1]{\mathfrak{#1}}
        \renewcommand{\it}[1]{\textit{#1}}
        \renewcommand{\bf}[1]{\textbf{#1}}
        \renewcommand{\sf}[1]{\textsf{#1}}
        \renewcommand{\phi}{\varphi}
        \renewcommand{\epsilon}{\varepsilon}

    % Meta
        \newcommand{\blob}{\bullet}
        \newcommand{\slot}{-}
        \newcommand{\cref}[1]{\tag{$\,#1\,$}}
        \newcommand{\qedin}{\tag*{$\blacksquare$}}
        \newcommand{\exqedin}{\tag*{$\blacklozenge$}}
        \renewcommand{\l}{\left}
        \renewcommand{\r}{\right}
        \renewcommand{\qed}{\phantom\qedhere\hfill$\blacksquare$}
        \renewcommand{\ref}[1]{\l(\,#1\,\r)}


\begin{document}
    \title{Tree-like Graphings of Countable Borel Equivalence Relations}
    \author{Zhaoshen Zhai}
    \date{\today}

    \begin{abstract}
        We present a streamlined exposition of a construction presented recently by R. Chen, A. Poulin, R. Tao, and A. Tserunyan, where it is proven that every locally-finite Borel graph with each component a quasi-tree induces a canonical treeable equivalence relation. {\color{red}{Write some more details...}}
    \end{abstract}

    \maketitle

    \setcounter{section}{-1}
    \section{Introduction}

    The purpose of this note is to provide a streamlined proof of a particular case of a construction presented in \cite{CPTT23}, in order to better understand the general formalism developed therein. We attempt to make this note self-contained, but nevertheless urge the reader to refer to the original paper for more detailed discussions and some generalizations of the results we have selected to include here.

    \subsection{Treeings of equivalence relations}

    A \textit{countable Borel equivalence relation (CBER)} on a standard Borel space $X$ is a Borel equivalence relation $E\subseteq X^2$ with each class countable. We are interested in special types of \textit{graphings} on a CBER $E\subseteq X^2$, i.e. a Borel graph $G\subseteq X^2$ whose connectedness relation is precisely $E$. For instance, a graphing of $E$ such that each component is a tree is called a \textit{treeing} of $E$, and the CBERs that admit treeings are said to be \textit{treeable}. The main results of \cite{CPTT23} provide new sufficient criteria for treeability of certain classes of CBERs, and in particular, they prove the following

    \begin{mainTheorem}[\cite{CPTT23}*{Theorem 1.1}]\label{thm:treeing_quasi-trees}
        If a CBER admits a locally-finite graphing whose components are quasi-trees, then it is treeable.
    \end{mainTheorem}

    Recall that metric spaces $X$ and $Y$ are \textit{quasi-isometric} if they are isometric up to a bounded multiplicative and additive error, and $X$ is a \textit{quasi-tree} if it is quasi-isometric to a simplicial tree; see \cite{Gro93} and \cite{DK18}.

    \subsection{Outline of the proof}

    Roughly speaking, the existence of a quasi-isometry $G|C\to T_C$ to a simplicial tree $T_C$ for each component $C\subseteq X$ induces a collection $\mc{H}(C)$ of `cuts' (subsets $H\subseteq C$ with finite boundary such that both $H$ and $C\comp H$ are connected), which are `tree-like' in the sense that
    \begin{enumerate}
        \item[1.] $\mc{H}(C)$ is \textit{finitely-separating}: each pair $x,y\in C$ is separated by finitely-many $H\in\mc{H}(C)$, and
        \item[2.] $\mc{H}(C)$ is \textit{dense towards ends}: each end in $G|C$ has a neighborhood basis in $\mc{H}(C)$.
    \end{enumerate}
    By Condition (1), these cuts have the structure of a profinite pocset with non-trivial points isolated, which in turn provide exactly the data to construct a `median graph' whose vertices are `ultrafilters'\footnote{As in \cite{CPTT23}, we call them \textit{orientations} instead, to avoid confusion with the more standard notion; see Definition \ref{def:orientation}.} thereof. Condition (2) then ensures that this graph has finite `hyperplanes', which allows us to apply a Borel `cycle-cutting' algorithm and obtain a canonical spanning tree. Each step above can be done in a uniform way to each component $C\subseteq G$, giving us the desired treeing of the CBER.
    \begin{equation*}
        \textrm{Quasi-tree}
            \ \ \overset{\ref{sec:graphs_with_dense_family_of_cuts}}{\longrightarrow}\ \ 
        \begin{gathered}
            \textrm{Dense family} \\[-4pt]
            \textrm{of cuts}
        \end{gathered}
            \ \ \overset{\ref{sec:pocset_of_dense_family_of_cuts}}{\longrightarrow}\ \ 
        \begin{gathered}
            \textrm{Pocset w/} \\[-4pt]
            \textrm{non-triv. pts.} \\[-4pt]
            \textrm{isolated}
        \end{gathered}
            \ \ \overset{\ref{sec:the_dual_median_graph_of_a_pocset}}{\longrightarrow}\ \ 
        \begin{gathered}
            \textrm{Median graph} \\[-4pt]
            \textrm{w/ finite} \\[-4pt]
            \textrm{hyperplanes}
        \end{gathered}
            \ \ \overset{\ref{sec:spanning_trees_of_median_graphs_with_finite_hyperplanes}}{\longrightarrow}\ \ 
        \begin{gathered}
            \textrm{Canonical} \\[-4pt]
            \textrm{spanning tree}
        \end{gathered}
    \end{equation*}

    {\color{red}{Write some more stuff to tie things together...}}

    \begin{remark*}
        We follow \cite{CPTT23}*{Convention 2.7}, where for a family $\mc{H}\subseteq2^X$ of subsets of a fixed set $X$, we write $\mc{H}^\ast\coloneqq\mc{H}\comp\l\{\em,X\r\}$ for the \textit{non-trivial} elements of $\mc{H}$.
    \end{remark*}

    \section{Graphs with Dense Families of Cuts}\label{sec:graphs_with_dense_family_of_cuts}

    \subsection{Ends of graphs}

    Let $(X,G)$ be a connected locally-finite graph, which, in the context of Theorem \ref{thm:treeing_quasi-trees}, will stand for a single component of the locally-finite graphing of a CBER.

    \begin{definition}
        For a subset $A\subseteq X$, we let $\del_\mathsf{iv}A\coloneqq A\cap\Ball_1(\lnot A)$ be its \textit{inner vertex boundary}, $\del_\mathsf{ov}A\coloneqq\del_\mathsf{iv}(\lnot A)$ be its \textit{outer vertex boundary}, and let $\del_\mathsf{ie}A\coloneqq G\cap(\del_\mathsf{ov}A\times\del_\mathsf{iv}A)$ and $\del_\mathsf{oe}A\coloneqq\del_\mathsf{ie}(\lnot A)$ respectively be its \textit{inner} and \textit{outer edge boundaries}. Let $\del_\mathsf{v}A\coloneqq\del_\mathsf{iv}A\cup\del_\mathsf{ov}A$ be the \textit{(total) vertex boundary} of $A$.
    \end{definition}

    Let $\mc{H}_{\del<\infty}(X)\subseteq2^X$ be the Boolean algebra of all $A\subseteq X$ with finite vertex boundary, which we refer to as \textit{cuts} in $X$. Later, we will restrict to the connected and co-connected cuts $\mc{H}_{\del<\infty}(X)\cap\mc{H}_\textrm{conn}(X)$.

    \begin{definition}
        The \textit{end compactification} of $(X,G)$ is the Stone space $\widehat{X}$ of $\mc{H}_{\del<\infty}(X)$, whose non-principal ultrafilters are the \textit{ends} of $(X,G)$.
    \end{definition}

    We identify $X\into\widehat{X}$ via principal ultrafilter map $x\mapsto p_x$, so $\widehat{X}\comp X$ is the set of ends of $G$. By definition, $\widehat{X}$ admits a basis of clopen sets of the form $\widehat{A}\coloneqq\{p\in\widehat{X}\st A\in p\}$ for each $A\in\mc{H}_{\del<\infty}(X)$.

    \begin{definition}
        A family $\mc{H}\subseteq\mc{H}_{\del<\infty}(X)$ of cuts is \textit{dense towards ends} of $(X,G)$ if $\mc{H}$ contains a neighborhood basis for every end in $\widehat{X}\comp X$.

        In other words, $\mc{H}$ is dense towards ends if for every $p\in\widehat{X}\comp X$ and every (clopen) neighborhood $\widehat{A}\ni p$, where $A\in\mc{H}_{\del<\infty}(X)$, there is some $H\in\mc{H}$ with $p\in\widehat{H}\subseteq\widehat{A}$; it is useful to note that $\widehat{H}\subseteq\widehat{A}$ iff $H\subseteq A$.
    \end{definition}

    The goal of this section is to show that certain cuts $\mc{H}\subseteq\mc{H}_{\del<\infty}$ induced in a (locally-finite) quasi-tree is dense towards ends. It will also be important that these cuts be connected, in that witnesses to density can also be found in $\mc{H}\cap\mc{H}_\textrm{conn}$, consisting of those cuts that are connected and co-connected.

    \begin{lemma}\label{lem:connected_witness_to_density}
        If a subpocset $\mc{H}\subseteq\mc{H}_{\del<\infty}$ is dense towards ends, then there is a subpocset $\mc{H}'\subseteq\mc{H}_{\del<\infty}\cap\mc{H}_\mathrm{conn}$, which is also dense towards ends, such that every $H'\in\mc{H}'$ has $\del_\mathsf{ie}H'\subseteq\del_\mathsf{ie}H$ for some $H\in\mc{H}$.
    \end{lemma}
    \begin{proof}
        A first attempt is to let $\mc{H}'$ be the connected components $H'_0$ of elements in $\mc{H}$, but this fails since $\lnot H'_0$ is not necessarily connected. Instead, we further take a component of $\lnot H'_0$, whose complement clearly co-connected, and is connected since it consists of $H'_0$ and the other components of $\lnot H'_0$, each of which is connected to $H'_0$ via $\del_\mathsf{ie}H'_0$. Formally, we let
        \begin{equation*}
            \mc{H}'\coloneqq\l\{H'\subseteq X\st H\in\mc{H}\textrm{ and }H'_0\in H/G\textrm{ and }\lnot H'\in\lnot H'_0/G\r\},
        \end{equation*}
        where $H/G$ denotes the $G$-components of $H$. Clearly $\del_\mathsf{ie}H'\subseteq\del_\mathsf{ie}H'_0\subseteq\del_\mathsf{ie}H$, so it remains to show that $\mc{H}'$ is dense towards ends.

        Fix an end $p\in\widehat{X}\comp X$ with $p\in\widehat{A}$ for some $A\in\mc{H}_{\del<\infty}(X)$. Let $B\supseteq\del_\mathsf{v}A$ be finite connected, which can be obtained by adjoining paths between its components. Then $\lnot B\in p$ since $p$ is non-principal, so there is $H\in\mc{H}$ with $p\in\widehat{H}\subseteq\lnot\widehat{B}$. We now use the above recipe to find the desired $H'\in\mc{H}'$:
            \begin{enumerate}
                \item Since $H\in\mc{H}_{\del<\infty}(X)$, it has finitely-many connected components, so exactly one of them belongs to $p$, say $p\in\widehat{H}'_0\subseteq\widehat{H}$. Note that $B\subseteq\lnot H\subseteq\lnot H'_0$.
                \item Since $B$ is connected, there is a unique component $\lnot H'\subseteq\lnot H'_0$ containing $B$.
            \end{enumerate}
        Observe that $H'\in\mc{H}'$ and $p\in\widehat{H}'$. Lastly, since $H'$ is connected and is disjoint from $\del_\mathsf{v}A\subseteq B$, and since $H'\subseteq\lnot A$ would imply $\lnot H'\in p$, this forces $H'\subseteq A$, and hence $p\in\widehat{H}'\subseteq\widehat{A}$ as desired.
    \end{proof}

    \subsection{Dense cuts induced by quasi-trees}\label{sec:dense_cuts_induced_by_quasi-trees}

    If $X$ is a quasi-tree $-$ and thus does not have arbitrarily long cycles $-$ we expect that there is some finite bound $R<\infty$ such that the ends in $\widehat{X}\comp X$ are `limits' of cuts $\mc{H}_{\diam(\del)\leq R}(X)\subseteq\mc{H}_{\del<\infty}(X)$ with boundary diameter bounded by $R$. We show that this is indeed the case, in the sense that $\mc{H}_{\diam(\del)\leq R}(X)\cap\mc{H}_\mathrm{conn}(X)$ is dense towards ends of $(X,G)$.

    \begin{lemma}\label{lem:coarse_equivalence_controls_boundary_diameter}
        Let $f:(X,G)\to(Y,T)$ be a coarse-equivalence between connected graphs. For a fixed $H\in\mc{H}_{\del<\infty}(Y)$, $\diam(\del_\mathsf{v}f^{-1}(H))$ is uniformly bounded in terms of $\diam(\del_\mathsf{v}H)$.
    \end{lemma}
    \begin{proof}
        Since $f$ is bornologous, let $S<\infty$ be such that $xGx'$ implies $d(f(x),f(x'))\leq S$, so that for any $(x,x')\in\del_\mathsf{ie}f^{-1}(H)$, there is a path of length $\leq S$ between $f(x)\not\in H$ and $f(x')\in H$. Thus both $d(f(x),\del_\mathsf{v}H)$ and $d(f(x'),\del_\mathsf{v}H)$ are bounded by $S$, so $f(\del_\mathsf{v}f^{-1}(H))\subseteq\Ball_S(\del_\mathsf{v}H)$ and hence
        \begin{equation*}
            \diam(f(\del_\mathsf{v}f^{-1}(H)))\leq\diam(\del_\mathsf{v}H)+2S.
        \end{equation*}
        That $f$ is a coarse-\textit{equivalence} gives us a uniform bound of $\diam(\del_\mathsf{v}f^{-1}(H))$ in terms of $\diam(\del_\mathsf{v}H)$.
    \end{proof}

    In particular, if $\diam(\del_\mathsf{v}H)$ is itself also uniformly bounded, then so is $\diam(\del_\mathsf{v}f^{-1}(H))$.

    \begin{proposition}\label{prp:invariant_of_density_coarse_equivalence}
        The class of connected locally-finite graphs in which $\mc{H}_{\diam(\del)\leq R}$ is dense towards ends for some $R<\infty$ is invariant under coarse equivalence.
    \end{proposition}
    \begin{proof}
        Let $(X,G)$, $(Y,T)$ be connected locally-finite graphs, $f:X\to Y$ be a coarse equivalence with quasi-inverse $g:Y\to X$, and suppose $\mc{H}_{\diam(\del)\leq S}(Y)$ is dense towards ends for some $S<\infty$. By Lemma \ref{lem:coarse_equivalence_controls_boundary_diameter}, pick some $R<\infty$ so that for any $H\in\mc{H}_{\diam(\del)\leq S}(Y)$, we have $f^{-1}(H)\in\mc{H}_{\diam(\del)\leq R}(X)$.

        Fix an end $p\in\widehat{X}\comp X$ with $p\in\widehat{A}$ for some $A\in\mc{H}_{\del<\infty}(X)$. We need to find some $B\in\mc{H}_{\del<\infty}(Y)$ such that $\widehat{f}(p)\in\widehat{B}$ and $f^{-1}(B)\subseteq A$, for then $\widehat{f}(p)\in\widehat{H}$ for some $B\supseteq H\in\mc{H}_{\diam(\del)\leq S}(Y)$, and hence we have
        \begin{equation*}
            p\in\widehat{f^{-1}(H)}\subseteq\widehat{f^{-1}(B)}\subseteq\widehat{A}
        \end{equation*}
        with $f^{-1}(H)\in\mc{H}_{\diam(\del)\leq R}(X)$. For convenience, let $D<\infty$ be the uniform distance $d(1_X,g\circ f)$.

        To this end, note that $\widehat{f}(p)\in\widehat{B}$ iff $p\in\widehat{f^{-1}(B)}$. Since $p\in\widehat{A}$, the latter can occur if $|A\symdiff f^{-1}(B)|<\infty$, and so we need to find such a $B\in\mc{H}_{\del<\infty}(Y)$ with the additional property that $f^{-1}(B)\subseteq A$.
        \begin{leftbar}
            \textit{Attempt 1.} Set $B\coloneqq g^{-1}(A)\in\mc{H}_{\del<\infty}(Y)$. Then $f^{-1}(B)\subseteq\Ball_D(A)$ since if $(g\circ f)(x)\in A$, then
                \begin{equation*}
                    d(x,A)\leq d(x,(g\circ f)(x))\leq d(1_X,g\circ f)=D.
                \end{equation*}
            By local-finiteness of $G$, we see that $A\symdiff f^{-1}(B)=A\comp f^{-1}(B)$ is finite, as desired.
        \end{leftbar}
        However, it is \textit{not} the case that $f^{-1}(B)\subseteq A$. To remedy this, we `shrink' $A$ by $D$ to $A'$ so that $\Ball_D(A')\subseteq A$, and take $B\coloneqq g^{-1}(A')$ instead. Indeed, $A'\coloneqq\lnot\Ball_D(\lnot A)\subseteq A$ works, since $f^{-1}(B)\subseteq\Ball_D(A')$ as before, so $A'\symdiff f^{-1}(B)=A'\comp f^{-1}(B)$ is finite. Also, $A\symdiff A'$ is finite since $x\in A\symdiff A'$ iff $x\in A$ and $d(x,\lnot A)\leq D$, so $A\symdiff f^{-1}(B)$ is finite too. It remains to show that $\Ball_D(A')\subseteq A$, for then $f^{-1}(B)\subseteq A$ as desired.

        Indeed, if $y\in\Ball_D(A')$, then by the (reverse) triangle-inequality we have $d(y,\lnot A)\geq d(x,\lnot A)-d(x,y)$ for all $x\in A'$. But $d(x,\lnot A)>D$, strictly, so $d(y,\lnot A)>D-D=0$, and hence $y\in A$.
    \end{proof}

    \begin{corollary}\label{cor:density_of_bounded_diameter_boundary_cuts_quasi_tree}
        If $(X,G)$ is a locally-finite quasi-tree, then the subpocset $\mc{H}_{\diam(\del)\leq R}(X)\cap\mc{H}_\mathrm{conn}(X)$ is dense towards ends for some $R<\infty$.
    \end{corollary}
    \begin{proof}
        Observe that $\mc{H}_{\diam(\del)\leq2}(T)$ is dense towards ends for any tree $T$, so Proposition \ref{prp:invariant_of_density_coarse_equivalence} proves the density of $\mc{H}_{\diam(\del)\leq R}(X)$ for some $R<\infty$. By Lemma \ref{lem:connected_witness_to_density}, there is a subpocset $\mc{H}'\subseteq\mc{H}_{\del<\infty}(X)\cap\mc{H}_\textrm{conn}(X)$ dense towards ends such that for every $H'\in\mc{H}'$, we have $\del_\mathsf{ie}H'\subseteq\del_\mathsf{ie}H$ for some $H\in\mc{H}_{\diam(\del)\leq R}(X)$. Hence we have $H'\in\mc{H}_{\diam(\del)\leq R}(X)\cap\mc{H}_\textrm{conn}(X)$, so the result follows.
    \end{proof}

    \section{Pocsets of Dense Families of Cuts}\label{sec:pocset_of_dense_family_of_cuts}

    \subsection{Pocsets of cuts}\label{sec:pocsets_of_cuts}

    The family $\mc{H}_{\diam\del\leq R}(X)$ of cuts have the structure of a `profinite pocset', which we first study abstractly. We then deduce some properties of the pocset induced by a \textit{dense} family of cuts.

    \begin{definition}\label{def:profinite_pocset}
        A \textit{pocset} $(\mc{H},\leq,\lnot,0)$ is a poset $(\mc{H},\leq)$ equipped with an order-reversing involution $\lnot:\mc{H}\to\mc{H}$ and a least element $0\neq\lnot0$ such that $0$ is the only lower-bound of $H,\lnot H$ for every $H\in\mc{H}$. We call the elements in $\mc{H}$ \textit{half-spaces}.

        A \textit{profinite pocset} is a pocset $\mc{H}$ equipped with a compact topology making $\lnot$ continuous and is \textit{totally order-disconnected}, in the sense that if $H\not\leq K$, then there is a clopen upward-closed $U\subseteq\mc{H}$ with $H\in U\not\ni K$.
    \end{definition}

    \begin{remark}
        Such a topology is automatically Hausdorff and zero-dimensional, making it a Stone space.
    \end{remark}

    We are primarily interested in subpocsets of $(2^X,\subseteq,\lnot,\em)$, which is profinite if equipped with the product topology of the discrete space $2$. Indeed, $2^X$ admits a base of \textit{cylinder sets} $-$ which are finite intersections of sets of the form $\pi^{-1}_x(i)$ where $x\in X$, $i\in\l\{0,1\r\}$, and $\pi_x:2^X\to2$ is the projection $-$ making $\lnot$ continuous since cylinders are clopen. Finally, for $H\not\leq K$,  let $U$ be the upward-closure of a clopen neighborhood $U_0\ni H$ separating it from $K$, which is clopen since $\lnot U_0$ is a finite union of cylinders.

    The following proposition gives a sufficient criteria for subpocsets of $2^X$ to be profinite. We also show in this case that every non-trivial element $H\in\mc{H}^\ast$ is isolated, which will important in Section \ref{sec:the_dual_median_graph_of_a_pocset}.

    \begin{lemma}\label{lem:finitely-separating_non-trivial_isolated}
        Let $X$ be a set and $\mc{H}\subseteq2^X$ be a subpocset. If $\mc{H}$ is finitely-separating, then $\mc{H}\subseteq2^X$ is closed and every non-trivial element is isolated.
    \end{lemma}
    \begin{proof}
        It suffices to show that the limit points of $\mc{H}$ are trivial, so let $A\in2^X\comp\l\{\em,X\r\}$. Fix $x\in A\not\ni y$. Since $\mc{H}$ is finitely-separating, there are finitely-many $H\in\mc{H}$ with $x\in H\not\ni y$, and for each such $H\in\mc{H}\comp\l\{A\r\}$, we have either some $x_H\in A\comp H$ or $y_H\in H\comp A$. Let $U\subseteq2^X$ be the family of all subsets $B\subseteq X$ containing $x$ and each $x_H$ but not $y$ or any $y_H$.

        This is the desired neighborhood isolating $A\in U$. Indeed, it is (cl)open since it is the \textit{finite} intersection of cylinders prescribed by the $x_H$'s and $y_H$'s, and it is disjoint from $\mc{H}\comp\l\{A\r\}$ by construction.
    \end{proof}

    In the case when $\mc{H}\subseteq\mc{H}_{\del<\infty}$ is a subpocset of cuts in a graph $(X,G)$, we can deduce that $\mc{H}\cap\mc{H}_\textrm{conn}\subseteq2^X$ is closed. Hence, the above lemma refines to the following

    \begin{lemma}\label{lem:finitely-separating_non-trivial_isolated_connected}
        Let $\mc{H}\subseteq\mc{H}_{\del<\infty}$ be a subpocset of cuts. If $\mc{H}$ is finitely-separating, then $\mc{H}\cap\mc{H}_\mathrm{conn}$ is closed and every non-trivial element is isolated.
    \end{lemma}
    \begin{proof}
        It suffices to show that no $H\in\mc{H}^\ast$ is a limit point of $\mc{H}_\textrm{conn}(X)$, for then $\mc{H}\cap\mc{H}_\textrm{conn}(X)$ is closed, so fix $H\in\mc{H}^\ast$. Let $U\subseteq2^X$ be the family of all subsets $B\subseteq X$ containing $\del_\mathsf{iv}H$ and disjoint from $\del_\mathsf{ov}H$, which clearly contains $H$, and is clopen since $\del_\mathsf{v}H$ is finite.

        Suppose that there is some $B\in U\cap\mc{H}_\mathrm{conn}(X)$. Then $H\subseteq B$, since if $x\in H\cap\lnot B$, then there is a path in $\lnot B$ to some $x'\in\del_\mathsf{ov}H\subseteq\lnot B$, which passes through $\del_\mathsf{iv}H\subseteq B$, a contradiction. The converse is similar.
    \end{proof}

    \begin{lemma}\label{lem:finitely_separating_iff_on_boundary_of_finite}
        Let $\mc{H}\subseteq2^X$ be a subpocset in a connected graph $(X,G)$. If each $x\in X$ is on the boundary of finitely-many half-spaces, then $\mc{H}$ is finitely-separating. The converse holds too if $(X,G)$ is locally-finite.
    \end{lemma}
    \begin{proof}
        Any $H\in\mc{H}$ separating $x,y\in X$ separates some edge on any fixed path between $x$ and $y$, and there are only finitely-many such $H$ for each edge. If $(X,G)$ is locally-finite, then each $x\in X$ is separated from each of its finitely-many neighbors by finitely-many $H\in\mc{H}$.
    \end{proof}

    In particular, if $(X,G)$ is locally-finite, then $\mc{H}_{\diam(\del)\leq R}(X)$ for any fixed $R<\infty$ (see Section \ref{sec:dense_cuts_induced_by_quasi-trees}) is finitely-separating. Indeed, since the $R$-ball around any fixed $x\in X$ is finite, any $H\in\mc{H}_{\diam(\del)\leq R}(X)$ with $x\in\del_\mathsf{v}H$ is contained in said $R$-ball, so there are finitely-many such half-spaces.

    This example, in tandem with Corollary \ref{cor:density_of_bounded_diameter_boundary_cuts_quasi_tree}, gives us, for each component $C\subseteq X$ of a quasi-treeing $G$ of a CBER, a family of cuts $\mc{H}(C)\coloneqq\mc{H}_{\diam(\del)\leq R_C}(C)\cap\mc{H}_\textrm{conn}(C)$ which is finitely-separating and dense towards ends of $G|C$ for some $R_C<\infty$. We use the former condition in Section \ref{sec:the_dual_median_graph_of_a_pocset} to construct a `median graph' $\mc{M}(\mc{H}(C))$, and use the latter to provide restrictions on $\mc{H}(C)$, and hence on $\mc{M}(\mc{H}(C))$ too, in Section \ref{sec:finiteness_conditions_on_pocsets_dense_cuts}. The conditions on $\mc{M}(\mc{H}(C))$ then allows us to \textit{canonically} construct a spanning tree thereof; we do so in Section \ref{sec:cycle_cutting_algorithm}, and prove in Section {\color{red}{???}} that this is the desired treeing of the CBER.

    \subsection{Finiteness conditions on $\mc{H}$ induced by dense cuts}\label{sec:finiteness_conditions_on_pocsets_dense_cuts}

    Let $(X,G)$ be a connected locally-finite graph and fix a finitely-separating subpocset $\mc{H}\subseteq\mc{H}_{\del<\infty}(X)\cap\mc{H}_\textrm{conn}(X)$ of cuts.

    \begin{definition}\label{def:nested}
        Let $\mc{H}\subseteq2^X$ be a pocset. Two half-spaces $H,K\in\mc{H}$ are \textit{nested} if $\lnot^iH\cap\lnot^jK=\em$ for some $i,j\in\l\{0,1\r\}$, where $\lnot^0H\coloneqq H$ and $\lnot^1H\coloneqq\lnot H$. We say that $\mc{H}$ is \textit{nested} if every pair $H,K\in\mc{H}$ is nested.
    \end{definition}

    \begin{lemma}\label{lem:connected_cuts_non_nested_finitely_others}
        Each $H\in\mc{H}$ is non-nested with finitely-many $K\in\mc{H}$.
    \end{lemma}
    \begin{proof}
        Fix $H\in\mc{H}$ and let $K\in\mc{H}$ be non-nested with $H$. By connectedness, the non-empty sets $H\cap K$ and $\lnot H\cap K$ are joined by a path in $K$, so $\del_\mathsf{v}H\cap K\neq\em$; similarly, $\del_\mathsf{v}H\cap\lnot K\neq\em$. For each $x\in\del_\mathsf{v}H\cap K$ and $y\in\del_\mathsf{v}H\cap\lnot K$, any fixed path $p_{xy}$ between them contains some $z\in\del_\mathsf{v}K\cap p_{xy}$; thus, any $K\in\mc{H}$ non-nested with $H$ contains some $z\in\del_\mathsf{v}K\cap p_{xy}$.

        Then, since there are finitely-many such $x,y\in\del_\mathsf{v}H$, for each of which there are finitely-many $z\in p_{xy}$, for each of which there are finitely-many $K\in\mc{H}$ with $z\in\del_\mathsf{v}K$ (by Lemma \ref{lem:finitely_separating_iff_on_boundary_of_finite}, since $\mc{H}$ is finitely-separating), there can only be finitely-many $K\in\mc{H}$ non-nested with $H$.
    \end{proof}

    \begin{proposition}\label{prp:dense_cuts_induces_proper_walling}
        If $\mc{H}$ is dense towards ends, then each $\mc{H}$-block is finite and each $H\in\mc{H}^\ast$ has finitely-many successors $K\in\mc{H}^\ast$.
    \end{proposition}
    \begin{proof}
        
    \end{proof}

    \section{The Dual Median Graph of a Profinite Pocset}\label{sec:the_dual_median_graph_of_a_pocset}

    % \subsection{The dual median graph of a pocset}

    Let $\mc{H}$ be a profinite subpocset with every non-trivial element isolated; say, if $\mc{H}$ is finitely-separating, and in particular the cuts $\mc{H}_{\diam(\del)\leq R}(X)$ for some locally-finite connected graph $(X,G)$. We present a classical construction in geometric group theory (see \cite{Dun79}, \cite{Rol98}, \cite{Sag95}, and \cite{NR03} for other applications) of a `tree-like' graph associated to such a pocset.
    
    \begin{definition}\label{def:orientation}
        An \textit{orientation} on $\mc{H}$ is an upward-closed subset $U\subseteq\mc{H}$ containing exactly one of $H,\lnot H$ for each $H\in\mc{H}$. We let $\mc{U}(\mc{H})$ denote the set of all orientations on $\mc{H}$ and let $\mc{U}^\circ(\mc{H})$ denote the clopen ones.
    \end{definition}

    Intuitively, an orientation is a `maximally consistent' choice of half-spaces\footnote{This can be formalized by letting $\sim$ be the equivalence relation on $\mc{H}$ given by $H\sim\lnot H$. Letting $\del:\mc{H}\to\mc{H}/\!\!\sim$ denote the quotient map, orientations $U\subseteq\mc{H}$ then correspond precisely to sections $\phi:\mc{H}/\!\!\sim\,\to\mc{H}$ of $\del$ such that $\phi(\del H)\not\subseteq\lnot\phi(\del K)$ for every $H,K\in\mc{H}$; the latter condition rules out `orientations' of the form $\leftarrow\hspace{-4.15pt}|\,\,|\hspace{-4.15pt}\rightarrow$.}.

    \begin{example}
        Each $x\in X$ induces its \textit{principal orientation} $\widehat{x}\coloneqq\l\{H\in\mc{H}\st x\in H\r\}=\mc{H}\cap\pi^{-1}_x(1)$ $-$ which is clearly clopen in $\mc{H}$ $-$ and gives us a canonical map $X\to\mc{U}^\circ(\mc{H})$. However, this map is \textit{not necessarily} injective, and we call a fiber $[x]_\mc{H}\coloneqq\l\{y\in X\st\widehat{x}=\widehat{y}\r\}$ thereof an \textit{$\mc{H}$-block}.
    \end{example}

    The goal of this section is to canonically construct a graph whose vertices are clopen orientations on $\mc{H}$.

    \begin{definition}
        A \textit{median graph} is a connected graph $(X,G)$ such that for any $x,y,z\in X$, the intersection
        \begin{equation*}
            [x,y]\cap[y,z]\cap[x,z]
        \end{equation*}
        is a singleton, whose element $\l\langle x,y,z\r\rangle$ is called the \textit{median} of $x,y,z$. Thus we have a ternary median operation $\l\langle\cdot,\cdot,\cdot\r\rangle:X^3\to X$, and a \textit{median homomorphism} $f:(X,G)\to(Y,H)$ is a map preserving said operation.
    \end{definition}

    Some basic properties of median graphs are given in Section \ref{sec:convex_half-spaces_of_median_graphs}. For more comprehensive references and their general theory, see \cite{Rol98} and \cite{Bow22}.

    \begin{proposition}\label{prp:construction_of_dual_median_graph}
        Let $\mc{H}$ be a profinite pocset with every non-trivial element isolated. Then the graph $\mc{M}(\mc{H})$, whose vertices are clopen orientations $\mc{U}^\circ(\mc{H})$ and whose edges are pairs $\l\{U,V\r\}$ with $V=U\symdiff\l\{H,\lnot H\r\}$ for some $\subseteq$-minimal $H\in U\comp\l\{\em,X\r\}$, is a median graph with path metric $d(U,V)=|U\symdiff V|/2$ and medians
        \begin{equation*}
            \begin{aligned}
                \l\langle U,V,W\r\rangle&\coloneqq\l\{H\in\mc{H}\st H\textrm{ belongs to at least two of }U,V,W\r\} \\
                                        &=(U\cap V)\cup(V\cap W)\cup(U\cap W).
            \end{aligned}
        \end{equation*}
    \end{proposition}

    \begin{proof}
        First, $V\coloneqq U\symdiff\l\{H,\lnot H\r\}$ as above is clopen since $H,\lnot H\in\mc{H}^\ast$ are isolated (whence $\l\{H\r\},\l\{\lnot H\r\}$ are clopen), and it is an orientation by $\subseteq$-minimality of $H$. That $\mc{M}(\mc{H})$ is connected follows from the following claim by noting that $U\symdiff V\not\ni\em,X$ is clopen, so it is a compact set of isolated points, whence finite.
        \begin{center}
            \begin{minipage}{0.95\textwidth}
                \begin{claim*}[\cite{Sag95}*{Theorem 3.3}]
                    There is a path between $U,V$ iff $U\symdiff V$ is finite, in which case
                    \begin{equation*}
                        d(U,V)=|U\symdiff V|/2=|U\comp V|=|V\comp U|.
                    \end{equation*}
                \end{claim*}
                \begin{proof}
                    If $(U_i)_{i<n}$ is a path from $U\eqqcolon U_0$ to $V\eqqcolon U_{n-1}$, then, letting $\l\{H_i,\lnot H_i\r\}\coloneqq U_i\symdiff U_{i-1}$ for all $1\leq i<n$ gives us a sequence $(H_i)_{i<n}$ inducing\footnote{In the sense that $U_i=U_{i-1}\symdiff\l\{H_i,\lnot H_i\r\}$ and $H_i\in U_i$ for each $1\leq i<n$; see \cite{Tse20}*{Definition 2.20}.} this path, whence $U\symdiff V$ consists of $\l\{H_i\r\}_{i<n}$ and their complements. Thus $U\symdiff V=2n=2d(U,V)$, as desired.

                    \hspace{0.2in}Conversely, if $U\symdiff V=\l\{H_1,\dots,H_n\r\}\sqcup\l\{K_1,\dots,K_m\r\}$ with $U\comp V=\l\{H_i\r\}$ and $V\comp U=\l\{K_j\r\}$, then $\lnot H_i\in V\comp U$ and $\lnot K_j\in U\comp V$ for all $i<n$ and $j<m$, so $n=m$ and $V=U\cup\l\{\lnot H_i\r\}\comp\l\{H_i\r\}$. We claim that there is a permutation $\sigma\in S_n$ such that $(H_{\sigma(i)})$ induces a path from $U\eqqcolon U_0$, which is the desired path from $U$ to $V$. Choose a minimal $H\in\l\{H_i\r\}$, which is also minimal in $U$: if $K\subseteq H$ for some $K\in U$, then $\lnot H\subseteq\lnot K$, and hence $\lnot K\in V$, so $K=H_i\subseteq H$ for some $i$, forcing $K=H$. Set $U_1\coloneqq U\symdiff\l\{H,\lnot H\r\}$, which is a clopen orientation. Continuing in this manner by choosing a minimal element in $\l\{H_i\r\}\comp\l\{H\r\}$ $-$ and so on $-$ gives us the desired path with $d(U,V)=n$.\phantom\qedhere\hfill$\square$
                \end{proof}
            \end{minipage}
        \end{center}
        Finally, we show that $\mc{M}(\mc{H})$ is a median graph. Fix $U,V,W\in\mc{U}^\circ(\mc{H})$, and note that for any $M\in\mc{U}^\circ(\mc{H})$, we have by the triangle inequality that $M\in[U,V]$ iff $(U\comp M)\cup(M\comp V)\subseteq U\comp V$, which clearly occurs iff $U\cap V\subseteq M\subseteq U\cup V$. Thus, a vertex $M$ lies in the triple intersection $[U,V]\cap[V,W]\cap[U,W]$ iff
        \begin{equation*}
            (U\cap V)\cup(V\cap W)\cup(U\cap W)\subseteq M\subseteq(U\cup V)\cap(V\cup W)\cap(U\cup W).
        \end{equation*}
        Note that the two sides coincide, so $M=\l\langle U,V,W\r\rangle$ $-$ which is clopen if $U,V,W$ are $-$ is as claimed.
    \end{proof}

    Given such a pocset $\mc{H}$, the graph $\mc{M}(\mc{H})$ constructed above is called the \textit{dual}\footnote{The name is justified by a Stone-type duality between median graphs with median homomorphisms and profinite pocsets whose non-trivial points are isolated with continuous maps, where from a median graph one can construct a canonical pocset of `convex' half-spaces (see \cite{CPTT23}*{Section 2.D} for details).} median graph of $\mc{H}$. An important special case of this construction is when $\mc{H}$ is \textit{nested}, in which case $\mc{M}(\mc{H})$ is a tree.

    \begin{lemma}
        Let $\mc{H}$ be a profinite pocset with non-trivial points isolated. If $\mc{H}$ is nested and $(H_i)_{i<n}\subseteq\mc{H}$ induces a path in $\mc{M}(\mc{H})$ with no backtracking, then $(H_i)$ is strictly increasing.
    \end{lemma}
    \begin{proof}
        Let $(U_i)_{i<n}$ be the induced path. By definition, we have $U_{i+1}=U_{i-1}\cup\{\lnot H_i,\lnot H_{i-1}\}\comp\l\{H_i,H_{i-1}\r\}$, so $H_i\neq\lnot H_{i-1}$, lest $U_{i+1}=U_{i-1}$. Thus, since $H_i\in U_i=U_{i-1}\cup\{\lnot H_{i-1}\}\comp\l\{H_{i-1}\r\}$, we see that $H_i\in U_{i-1}$. Clearly $H_i\neq H_{i-1}$. It suffices to remove the three cases when $\lnot H_i\subseteq H_{i-1}$, $H_{i-1}\subseteq\lnot H_i$, and $H_i\subseteq H_{i-1}$, since then nestedness of $\mc{H}$ gives us $H_{i-1}\subsetneq H_i$, as desired.

        Indeed, if $\lnot H_i\subseteq H_{i-1}$, then $H_{i-1}\in U_{i+1}$ by upward-closure of $U_{i+1}\ni\lnot H_i$. But since $H_{i-1}\neq\lnot H_i$, we have by definition of $U_{i+1}$ that $H_{i-1}\in U_i$, a contradiction. The other cases are similar.
    \end{proof}

    \begin{corollary}\label{cor:nested_implies_tree}
        If $\mc{H}$ is nested, then $\mc{M}(\mc{H})$ is a tree.
    \end{corollary}
    \begin{proof}
        A cycle in $\mc{M}(\mc{H})$ is induced by a \textit{strictly} increasing chain $H_0\subset\cdots\subset H_n\subset H_0$, which is absurd.
    \end{proof}

    Nonetheless, in the general non-nested case, the condition that $\mc{H}$ is dense towards ends (and in particular, the finiteness conditions on $\mc{H}$ that follows; see Section \ref{sec:finiteness_conditions_on_pocsets_dense_cuts}) gives us a sufficient criteria for the existence of a \textit{canonical} spanning tree of $\mc{M}(\mc{H})$, which we prove in the following section.

    \section{Median Graphs, Convex Half-spaces, and Canonical Spanning Trees}\label{sec:spanning_trees_of_median_graphs_with_finite_hyperplanes}

    Throughout this section, let $(X,G)$ be a median graph, which, in the context of Theorem \ref{thm:treeing_quasi-trees}, is the dual median graph of the pocset $\mc{H}_{\diam(\del)\leq R}$ given by the quasi-isometry.

    The presence of the median operation gives rise to a canonical finitely-separating pocset of \textit{convex} (and co-convex) half-spaces $\mc{H}_\textrm{cvx}(X)\subseteq2^X$ with a very rigid structure. To study them, we need to first understand the geometry of \textit{projections} therein.

    \subsection{Projections in median graphs}\label{sec:projections_in_median_graphs}

    \renewcommand{\-}{\textrm{---}}

    For vertices $x,y,z\in X$, we write $x\-y\-z$ for $y\in[x,z]$. By the triangle inequality, we have for all $w,x,y,z\in X$ that
    \begin{equation*}
        (w\-x\-y\textrm{ and }w\-y\-z)\ \ \ \ \Leftrightarrow\ \ \ \ (w\-x\-z\textrm{ and }x\-y\-z),
    \end{equation*}
    and both sides occur iff there is a geodesic from $w$ to $x$ to $y$ to $z$, which we write as $w\-x\-y\-z$.

    \begin{lemma}\label{lem:projections}
        For any $\em\neq A\subseteq X$ and $x\in X$, there is a unique point in $\cvx(A)$ between $x$ and every point in $A$, called the $\mathrm{projection}$ of $x$ towards $A$, denoted $\proj_A(x)$.

        Moreover, we have $\bigcap_{a\in A}[x,a]=[x,\proj_A(x)]$, and for any $y$ in this set, we have $\proj_A(y)=\proj_A(x)$.
    \end{lemma}
    \begin{proof}
        To show existence, pick any $a_0\in A$. Given $a_n\in\cvx(A)$, if there exists $a\in A$ with $a_n\not\in[x,a]$, set $a_{n+1}\coloneqq\l\langle x,a,a_n\r\rangle\in\cvx(A)$. Then $a_0\-a_1\-\cdots\-a_n\-x$ for all $n$, so this sequence terminates in at most $d(a_0,x)$ steps at a point in $\cvx(A)$ between $x$ and every point in $A$. For uniqueness, if there exist two such points $a,b\in\cvx(A)$, then $x\-a\-b$ and $x\-b\-a$, forcing $a=b$.

        Finally, if $x\-y\-\proj_A(x)$ and $a\in A$, then $x\-\proj_A(x)\-a$ and hence $x\-y\-a$. Conversely, let $x\-y\-a$ for all $a\in A$. Since $[y,a]\subseteq[x,a]$ for all $a$, we see that
        \begin{equation*}
            \proj_A(y)\in\cvx(A)\cap\bigcap_{a\in A}[y,a]\subseteq\cvx(A)\cap\bigcap_{a\in A}[x,a]
        \end{equation*}
        and hence $\proj_A(y)=\proj_A(x)$ by uniqueness. But since $y\-\proj_A(y)\-a$, we have $x\-y\-\proj_A(y)$, and hence $x\-y\-\proj_A(x)$ as desired.
    \end{proof}

    \begin{remark}\label{rem:projections}
        It follows from the proof above that for any median homomorphism $f:(X,G)\to(Y,H)$, we have $f(\proj_A(x))=\proj_{f(A)}(f(x))$ for any $\em\neq A\subseteq X$ and $x\in X$. Indeed, we have
        \begin{equation*}
            \proj_A(x)=\l\langle x,a_m,\dots,\l\langle x,a_2,\l\langle x,a_1,a_0\r\rangle\r\rangle\dots\r\rangle
        \end{equation*}
        for some $m\leq d(a_0,x)$ and $a_0,\dots,a_m\in A$, and this is preserved by $f$.
    
        For $A\coloneqq\l\{a,b\r\}$, we have $\proj_A(x)=\l\langle a,b,x\r\rangle$, and hence $\cvx(A)=\proj_A(X)=\l\langle a,b,X\r\rangle=[a,b]$.
    \end{remark}

    \begin{lemma}\label{lem:cones_are_convex}
        For each $x,y\in X$, $\cone_x(y)$ is convex, and if $xGy$, then $\cone_x(y)\sqcup\cone_y(x)=X$.
    \end{lemma}
    \begin{proof}
        Fix $a,b\in\cone_x(y)$ and $a\-c\-b$. It suffices to show that $x\-y\-\l\langle a,c,x\r\rangle$, for then $x\-y\-c$ since we have $x\-\l\langle a,c,x\r\rangle\-c$. Indeed, it follows from the following observations.
        \begin{itemize}
            \item $x\-y\-\l\langle a,b,x\r\rangle$, since $\l\langle a,b,x\r\rangle=\proj_{\l\{a,b\r\}}(x)$ and so $[x,\l\langle a,b,x\r\rangle]=[x,a]\cap[x,b]\ni y$ by Lemma \ref{lem:projections}.
            \item $x\-\l\langle a,b,x\r\rangle\-\l\langle a,c,x\r\rangle$, which follows from $\l\langle a,b,x\r\rangle\-\l\langle a,c,x\r\rangle\-a$, since $x\-\l\langle a,b,x\r\rangle\-a$ by definition. Indeed, we have $\l\langle a,c,x\r\rangle$ is in both $[a,x]$ and $[a,c]\subseteq[a,b]$, and since $\proj_{\l\{b,x\r\}}(a)=\l\langle a,b,x\r\rangle$, we have again by Lemma \ref{lem:projections} that $[\l\langle a,b,x\r\rangle,a]=[a,x]\cap[a,b]\ni\l\langle a,c,x\r\rangle$.
        \end{itemize}
        Finally, take $z\in X$ and consider $w\coloneqq\l\langle x,y,z\r\rangle\subseteq[x,y]$. Either $w=x$ or $w=y$ (but not both), giving us the desired partition.
    \end{proof}

    In particular, this shows that if $xGy$, then $\cone_x(y)\in\mc{H}^\ast_\textrm{cvx}(X)$. The convexity of cones also shows, in the situation of Lemma \ref{lem:projections}, that $\proj_A=\proj_{\cvx(A)}$, i.e., $\proj_A(x)$ is also between $x$ and every point in $\cvx(A)$. Indeed, note that $\cone_x(\proj_A(x))$ is convex and contains $A$, so it contains $\cvx(A)$ too.

    \begin{lemma}\label{lem:projection_homomorphism}
        $\proj_A:X\onto\cvx(A)$ is a median homomorphism with $\proj_A\circ\cvx=\cvx\circ\proj_A$.
    \end{lemma}
    \begin{proof}
        The second claim follows from the first since, by Remark \ref{rem:projections}, we have
        \begin{equation*}
            f(\cvx(B))=f(\proj_B(X))=\proj_{f(B)}(f(X))=\cvx(f(B))
        \end{equation*}
        for all median homomorphisms $f:X\onto Y$ and $B\subseteq X$, so it in particular applies to $f\coloneqq\proj_A$.

        To this end, let $x\-y\-z\in X$ and set $w\coloneqq\l\langle\proj_A(x),\proj_A(y),\proj_A(z)\r\rangle\in\cvx(A)$. It suffices to show that $y\-w\-a$ for all $a\in A$, for then $w=\proj_A(y)$ and hence $\proj_A(x)\-\proj_A(y)\-\proj_A(z)$. But we have $y\-\proj_A(y)\-a$ already, so it further suffices to show that $y\-w\-\proj_A(y)$. For this, we note that
        \begin{equation*}
            x\-\proj_A(x)\-\proj_A(y)\ \ \ \ \textrm{and}\ \ \ \ \proj_A(x)\-w\-\proj_A(y),
        \end{equation*}
        so $x\-w\-\proj_A(y)$, and similarly $z\-w\-\proj_A(y)$. Thus, it follows that
        \begin{equation*}
            \begin{aligned}
                w\in[\proj_A(y),x]\cap[\proj_A(y),z]&=[\proj_A(y),\proj_{\l\{x,z\r\}}(\proj_A(y))]\ \ \ \ && \textrm{Lemma \ref{lem:projections}} \\
                                                    &=[\proj_A(y),\proj_{\l[x,z\r]}(\proj_A(y))] \\
                                                    &\subseteq[\proj_A(y),y],\ \ \ \ && \textrm{Lemma \ref{lem:projections}}
            \end{aligned}
        \end{equation*}
        where the second equality follows from $\cvx(\l\{x,z\r\})=[x,z]$, and hence $\proj_{\l\{x,z\r\}}=\proj_{\l[x,z\r]}$.
    \end{proof}

    \subsection{Convex half-spaces of median graphs}\label{sec:convex_half-spaces_of_median_graphs}

    We now use projections to explore the geometry of convex half-spaces in median graphs. In particular, they are very rigid:

    \begin{proposition}\label{prp:half-spaces_are_cones}
        Each edge $(x,y)\in G$ is on a unique hyperplane, namely the inward boundary of $\cone_x(y)$, and conversely, each half-space $H\in\mc{H}^\ast_\mathrm{cvx}(X)$ is $\cone_x(y)$ for every $(x,y)\in\del_\mathsf{ie}H$.

        Thus, hyperplanes are equivalence classes of edges. Furthermore, this equivalence relation is generated by parallel sides of squares (i.e., $4$-cycles).
    \end{proposition}
    \begin{proof}
        We have $\cone_x(y)\in\mc{H}_\textrm{cvx}^\ast(X)$ by the above lemma, and clearly $(x,y)\in\del_\mathsf{ie}\cone_x(y)$. Conversely, take $H\in\mc{H}_\textrm{cvx}^\ast(X)$ and any $(x,y)\in\del_\mathsf{ie}H$. Then $H=\cone_x(y)$, for if $z\in H\cap\lnot\cone_x(y)$, then $z\in\cone_y(x)$, and hence $x\in[y,z]\subseteq H$ by convexity of $H$, a contradiction; if $z\in\cone_x(y)\cap\lnot H$, then $[x,z]\subseteq\lnot H$ by convexity of $\lnot H$, and hence $y\not\in H$, a contradiction.

        Finally, parallel edges of a strip of squares generate the same hyperplane since, for a given square, each vertex is between its neighbors and hence any hyperplane containing an edge contains its opposite edge. On the other hand, let $(a,b),(c,d)\in\del_\mathsf{ie}H$ for some $H\in\mc{H}_\textrm{cvx}^\ast(X)$. Note that $\del_\mathsf{ov}H=\proj_{\lnot H}(H)$ is convex since $H$ is, and $\proj_{\lnot H}$ preserves convexity by Proposition \ref{lem:projection_homomorphism}, so any geodesic between $a,c\in\del_\mathsf{ov}H$ lies in $\del_\mathsf{ov}H$. Matching this geodesic via $\del_\mathsf{ie}H:\del_\mathsf{ov}H\to\del_\mathsf{iv}H$ gives us a geodesic between $b,d$ in $\del_\mathsf{iv}H$, which together with the matching forms the desired strip of squares.
    \end{proof}

    \begin{corollary}\label{cor:non-nested_iff_embedding_of_hamming}
        Two half-spaces $H,K\in\mc{H}_\mathrm{cvx}^\ast(X)$ are non-nested iff there is an embedding $\l\{0,1\r\}^2\into X$ of the Hamming cube into the four corners $\lnot^iH\cap\lnot^jK$.

        In particular, if $H,K\in\mc{H}_\mathrm{cvx}^\ast(X)$ are non-nested, then $\del_\mathsf{v}H\cap\del_\mathsf{v}K\neq\em$.
    \end{corollary}
    \begin{proof}
        Let $H,K$ be non-nested and take $x_1\in H\cap K$ and $x_2\in H\cap\lnot K$. Since $H$ is connected, any geodesic between $x_1,x_2$ crosses an edge $(x_1',x_2')\in\del_\mathsf{oe}K$ in $H$. Similarly, there is an edge $(y_1',y_2')\in\del_\mathsf{oe}K$ in $\lnot H$, so we may slide both edges along $\del_\mathsf{oe}K$ to obtain the desired square (see Proposition \ref{prp:half-spaces_are_cones}).

        Conversely, the half-spaces cutting the square are clearly non-nested.
    \end{proof}

    \begin{lemma}[Helly]\label{lem:helly}
        Any finite intersection of pairwise-intersecting non-empty convex sets is non-empty.
    \end{lemma}
    \begin{proof}
        For pairwise-intersecting convex sets $H_1,H_2,H_3$, pick any $x\in H_1\cap H_2$, $y\in H_1\cap H_3$ and $z\in H_2\cap H_3$; their median $\l\langle x,y,z\r\rangle$ then lies in $H_1\cap H_2\cap H_3$.

        Suppose that it holds for some $n\geq 3$ and let $H_1,\dots,H_{n+1}\subseteq X$ pairwise-intersect. Then $\l\{H_i\cap H_{n+1}\r\}_{i\leq n}$ is a family of $n$ pairwise-intersecting convex sets, so $\bigcap_{i\leq n+1}H_i=\bigcap_{i\leq n}(H_i\cap H_{n+1})$ is non-empty.
    \end{proof}

    Lastly, we have some useful finiteness conditions on convex half-spaces; the former implies that $\mc{H}_\textrm{cvx}(X)$ is finitely-separating, and the latter allows us to replace finite sets with their convex hulls.

    \begin{lemma}\label{lem:half_space_separating_convex}
        Any two disjoint convex sets $\em\neq A,B\subseteq X$ can be separated by a half-space $A\subseteq H\subseteq\lnot B$, and furthermore we have $d(A,B)=|\!\l\{H\in\mc{H}_\mathrm{cvx}(X)\st A\subseteq H\subseteq\lnot B\r\}\!|$.
    \end{lemma}
    \begin{proof}
        Pick a geodesic $A\ni x_0Gx_1G\cdots Gx_n\in B$, where $n\coloneqq d(A,B)$. Then $H\coloneqq\cone_{x_1}(x_0)$, which is a half-space by Lemma \ref{lem:cones_are_convex}, separates $A,B$ since $x_0=\proj_A(x_n)$, and thus we have $A\subseteq\cone_{x_n}(x_0)\subseteq\cone_{x_1}(x_0)$ and $B\subseteq\cone_{x_0}(x_n)\subseteq\cone_{x_0}(x_1)$.

        Moreover, each such half-space $A\subseteq H\subseteq\lnot B$ satisfies $x_i\in H\not\ni x_{i+1}$ for a unique $i<n$, and conversely each pair $(x_i,x_{i+1})$ has a unique half-space separating them, so we have the desired bijection.
    \end{proof}

    \begin{lemma}\label{cor:convex_of_finite_is_finite}
        Every interval $[x,y]$ is finite. More generally, if $A\subseteq X$ is finite, then so is $\cvx(A)$.
    \end{lemma}
    \begin{proof}
        The singletons $\l\{x\r\}$ and $\l\{y\r\}$ are convex, so there are finitely-many half-spaces $H\subseteq[x,y]$. But each $z\in[x,y]$ is determined uniquely by those half-spaces containing it, so $[x,y]$ is finite.

        Let $A\coloneqq\l\{x_0,\dots,x_n\r\}$. Since $\cvx(A)=\proj_A(X)$, we have by Remark \ref{rem:projections} that points in $\cvx(A)$ are of the form $\l\langle x,x_n,\dots,\l\langle x,x_2,\l\langle x,x_1,x_0\r\rangle\r\rangle\r\rangle$, which is finite by induction using that intervals are finite.
    \end{proof}

    \subsection{Canonical spanning trees}\label{sec:cycle_cutting_algorithm}

    We now present the Borel cycle-cutting algorithm that can be preformed canonically on any countable median graph with finite hyperplanes. We do so by colouring the half-spaces $\mc{H}^\ast_\textrm{cvx}(X)$ into certain nested half-spaces $\mc{H}_n\subseteq\mc{H}_\textrm{cvx}(X)$, from which we inductively build a spanning forest by leveraging a tree structure on the $\mc{H}_n$-blocks $X/\mc{H}_n$. This tree is constructed as follows.

    \begin{lemma}\label{lem:H_blocks_form_a_median_graph}
        For any subpocset $\mc{H}\subseteq\mc{H}_\mathrm{cvx}(X)$, the principal orientations map $X\to\mc{U}^\circ(\mc{H})$ is surjective.
    \end{lemma}
    \begin{proof}
        Let $U\in\mc{U}^\circ(\mc{H})$. Since $U\subseteq\mc{H}$ is clopen, there is a finite set $A\subseteq X$ $-$ which we may assume to be convex by Corollary \ref{cor:convex_of_finite_is_finite} $-$ such that for all $H\in\mc{H}$, we have $H\in U$ iff there is $K\in U$ with $H\cap A=K\cap A$. Note that $K\cap A\neq\em$ for every $K\in U$, since otherwise $\em\in U$. Furthermore, $H\cap K\neq\em$ for every $H,K\in U$, since otherwise we have $H\subseteq\lnot K$, and so $\lnot K\in U$. By Lemma \ref{lem:helly}, we have $(H\cap A)\cap(K\cap A)=H\cap K\cap A\neq\em$, and applying it again furnishes some $x\in\bigcap_{H\in U}H\cap A$ in $X$; note that the latter intersection is finite. Thus $U\subseteq\widehat{x}$, so $U=\widehat{x}$ since both are orientations.
    \end{proof}

    This lemma is crucial, in that it induces an isomorphism $X/\mc{H}\iso\mc{M}(\mc{H})$. Indeed, $\mc{H}$ is finitely-separating by Lemma \ref{lem:finitely_separating_iff_on_boundary_of_finite} and Proposition \ref{prp:half-spaces_are_cones}, and so $\mc{U}^\circ(\mc{H})$ is the vertices of the median graph $\mc{M}(\mc{H})$ by Lemma \ref{lem:finitely-separating_non-trivial_isolated} and Proposition \ref{prp:construction_of_dual_median_graph}. The fibers of $X\to\mc{U}^\circ(\mc{H})$ are the $\mc{H}$-blocks, so we have the desired bijection.

    Explicitly, two $\mc{H}$-blocks $([x]_\mc{H},[y]_\mc{H})$ are said to be \textit{$G$-adjacent} if $(\widehat{x},\widehat{y})\in\mc{M}(\mc{H})$.

    \begin{lemma}\label{lem:finite_hyperplane_iff_non-nested_with_finitely-many}
        A median graph $(X,G)$ has finite hyperplanes iff each half-space $H\in\mc{H}_\mathrm{cvx}(X)$ is non-nested with finitely-many others.
    \end{lemma}
    \begin{proof}
        By Corollary \ref{cor:non-nested_iff_embedding_of_hamming}, a half-space $K$ non-nested with $H$ corresponds uniquely to a pair $(x,y)\in\del_\mathsf{ie}H$ with $x,y\in\del_\mathsf{iv}K$, so there are finitely-many such half-spaces $K$ iff $\del_\mathsf{ie}H$ is finite.
    \end{proof}

    \begin{proposition}\label{prp:canonical_spanning_trees}
        Every countable median graph with finite hyperplanes admits a canonical spanning tree.
    \end{proposition}
    \begin{proof}
        Let $(X,G)$ be a countable median graph with finite hyperplanes. By Corollary \ref{cor:non-nested_iff_embedding_of_hamming}, if two half-spaces $H,K\in\mc{H}^\ast_\textrm{cvx}(X)$ are non-nested, then $\del_\mathsf{v}H\cap\del_\mathsf{v}K\neq\em$, so looking at the intersection graph of the boundaries furnishes a countable colouring $\mc{H}^\ast_\textrm{cvx}(X)=\bigsqcup_{n\in\N}\mc{H}^\ast_n$ such that each $H,\lnot H$ receive the same color and that each $\mc{H}_n\coloneqq\mc{H}_n^\ast\cup\l\{\em,X\r\}$ is a \textit{nested} subpocset. For each $n\in\N$, let $\mc{K}_n\coloneqq\bigcup_{m\geq n}\mc{H}_m$.

        We shall inductively construct an increasing chain of subforests $T_n\subseteq G$ such that the components of $T_n$ are exactly the $\mc{K}_n$-blocks. Then, the increasing union $T\coloneqq\bigcup_nT_n$ is a spanning tree, since each $(x,y)\in G$ lies in a $\mc{K}_n$-block for sufficiently large $n$ (namely, the $n$ such that $\cone_x(y)\in\mc{H}^\ast_{n-1}$, since $\cone_x(y)$ and its complement are the only half-spaces separating $x$ and $y$ by Proposition \ref{prp:half-spaces_are_cones}).

        Since each pair of distinct points is separated by a half-space, the $\mc{K}_0=\mc{H}_\textrm{cvx}(X)$-blocks are singletons, so put $T_0\coloneqq\em$. Suppose that a forest $T_n$ is constructed as required. Note that each $\mc{K}_{n+1}$-block $Y\in X/\mc{K}_{n+1}$ is not separated by any half-spaces in $\mc{H}_m$ for $m>n$, but is separated (by Proposition \ref{prp:half-spaces_are_cones}) by $\mc{H}_n$ into the $\mc{K}_n$-blocks contained in $Y$, and those correspond precisely to the $\mc{H}_n$-blocks in $Y/\mc{H}_n$. For each $G$-adjacent pair $A,B\in Y/\mc{H}_n$, which is separated by a unique half-space $H\in\mc{H}_n$ by Lemma \ref{lem:half_space_separating_convex}, we may pick an edge from the \textit{finite} hyperplane $\del_\mathsf{ie}H$. Since each $Y/\mc{H}_n$ is a tree by Corollary \ref{cor:nested_implies_tree}, and a single edge is picked between every pair of $G$-adjacent blocks therein, the graph $T_{n+1}$ obtained from $T_n$ by adding all such edges is a forest whose components are exactly the $\mc{K}_{n+1}$-blocks.
    \end{proof}

    To apply this proposition to the dual median graph $\mc{M}(\mc{H})$ of a subpocset $\mc{H}\subseteq\mc{H}_{\del<\infty}\cap\mc{H}_\textrm{conn}$ of finitely-separating cuts, we need to show that each half-space in $\mc{H}_\mathrm{cvx}(\mc{M}(\mc{H}))$ is non-nested with finitely-many others and invoke Lemma \ref{lem:finite_hyperplane_iff_non-nested_with_finitely-many}. By Lemma \ref{lem:connected_cuts_non_nested_finitely_others}, the following suffices.
    
    \begin{lemma}
        Fix a half-space $K\in\mc{H}_\mathrm{cvx}(\mc{M}(\mc{H}))$. Then there is a unique $H\in\mc{H}$ such that
        \begin{equation*}
            K=\widehat{H}\coloneqq\l\{W\in\mc{U}^\circ(\mc{H})\st W\in H\r\},
        \end{equation*}
        and if $K'\in\mc{H}_\mathrm{cvx}(\mc{M}(\mc{H}))$ is non-nested with $K$, say with $K'=\widehat{H}'$, then $H'$ is non-nested with $H$.
    \end{lemma}
    \begin{proof}
        
    \end{proof}

    \begin{bibdiv}
        \begin{biblist}*{labels={alphabetic}}
            \bibselect{setup/bibliography}
        \end{biblist}
    \end{bibdiv}
\end{document}
