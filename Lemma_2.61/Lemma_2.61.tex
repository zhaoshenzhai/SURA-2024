\documentclass{amsart}
\usepackage{amsfonts, amsmath, amssymb, amsthm, amsrefs}
\usepackage{tikz, tikz-3dplot, tikzpagenodes}
\usepackage{graphicx, xcolor, geometry, mdframed}
\usepackage{enumitem, nicematrix}
\usepackage{mathtools, mathrsfs}
\usepackage{caption, subcaption}
\usepackage{bookmark, xifthen}
\usepackage{transparent}

\usetikzlibrary{matrix, positioning, patterns, decorations.markings, arrows, arrows.meta, backgrounds, math, cd}
\tikzset{->-/.style={decoration={ markings, mark=at position #1 with {\arrow{>}}},postaction={decorate}}}

\definecolor{darkBlue}{RGB}{0, 0, 138}
\hypersetup{colorlinks=true, allcolors=darkBlue}
\theoremstyle{definition}\newtheorem{question}{\color{red}Question}

\input{macros.sty}

\begin{document}
    \title{Questions in Lemma 2.61}
    \author{Zhaoshen Zhai}
    \date{\today}
    \maketitle

    \section{Results from the Paper}

    \setcounter{section}{2}
    I'll first reference some previous results, and then reproduce Lemma \ref{lem:2.61} and its proof here.

    \setcounter{theorem}{33}
    \begin{corollary}\label{cor:2.34}
        Every interval $[x,y]$ in a median graph is finite. More generally, a convex hull $\cvx(\l\{x_0,\dots,x_n\r\})$ of finitely many points is finite.
    \end{corollary}

    \setcounter{theorem}{57}
    \begin{lemma}\label{lem:2.58}
        For a median graph $(X,G)$ with finite hyperplanes, $\mc{H}_{\del<\infty}(X)\subseteq2^X$ is the Boolean subalgebra generated by $\mc{H}_\textrm{cvx}(X)$.
    \end{lemma}

    \setcounter{theorem}{60}
    \begin{lemma}\label{lem:2.61}
        For a locally-finite median graph $(X,G)$ with finite hyperplanes, each end has a neighborhood basis of half-spaces.
    \end{lemma}
    \begin{proof}
        Let $U\in\widehat{X}\comp X$ be an end, $\widehat{A}\ni U$ be a clopen neighborhood where $A\in\mc{H}_{\del<\infty}(X)$. By Corollary \ref{cor:2.34}, $\cvx(\del_\sf{ov}A)\subseteq X$ is finite. By Lemma \ref{lem:2.58} (applied to it and each point on its outer boundary), it is a finite intersection of half-spaces. One such half-space $H$ must not contain $U$, since {\color{red}{$U\not\in\cvx(\del_\sf{ov}A)$}}\footnote{I think this should be $\cvx(\del_\sf{ov}A)\not\in U$.}. Then $U\in\lnot\widehat{H}$, whence $A\cap\lnot H\neq\em$, whence $\lnot H\subseteq A$ since $\del_\sf{ov}A\cap\lnot H=\em$.
    \end{proof}

    \setcounter{section}{1}
    \section{Discussion and Questions}

    I'll give some more details in the proof. Let $U\in\widehat{A}$ be as above.

    \begin{proof}
        ... $Y\coloneqq\cvx(\del_\sf{ov}A)$ is finite, so there are $H_0,\dots,H_n\in\mc{H}_\textrm{cvx}(X)$ such that $Y=\bigcap_{i<n}H_i$. Then there is some $i<n$ such that $H\coloneqq H_i\not\in U$, since otherwise we have $Y\in U$, a contradiction since $Y$ is finite and $U$ is a \textit{non-principal} ultrafilter. Thus $\lnot H\in U$, so $U\in\lnot\widehat{H}$. Since $A\in U$, we have $A\cap\lnot H\neq\em$. We claim that $\lnot H\subseteq A$, so that $U\in\lnot\widehat{H}\subseteq\widehat{A}$ as desired.

        Indeed, fix some $x_0\in A\cap\lnot H$. If there is some $x\in\lnot A\cap\lnot H$, then the interval $[x_0,x]$ contains some $y\in\del_\sf{ov}A$. Since $\lnot H$ is convex, we see that $y\in[x_0,x]\subseteq\lnot H$, so that $y\in\del_\sf{ov}A\cap\lnot H$. This is absurd, since $\del_\sf{ov}A\subseteq Y\subseteq H$.
    \end{proof}

    \begin{question}
        Lemma \ref{lem:2.58} only guarantees that every $A\in\mc{H}_{\del<\infty}(X)$ is a finite \textit{boolean combination} of half-spaces in $X$, not finite \textit{intersections} thereof.

        Assuming that $Y=\bigcap_{i<n}H_i$, I think the rest of the proof applies also for non-locally-finite graphs, so somehow this should be justified by local-finiteness of $G$.
    \end{question}

    \begin{question}
        Why not set $Y\coloneqq\del_\sf{ov}A$ directly? That is, write $\del_\sf{ov}A$ as a finite intersection of half-spaces? The rest of the proof should work with this modification ($\del_\sf{ov}A$ is itself finite, so $\del_\sf{ov}A\not\in U$.). I don't see how convexity of $Y$ plays a role here.
    \end{question}

    \begin{question}
        
    \end{question}
\end{document}
