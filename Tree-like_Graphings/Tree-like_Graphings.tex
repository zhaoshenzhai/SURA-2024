\documentclass{amsart}
\usepackage{amsfonts, amsmath, amssymb, amsthm, amsrefs}
\usepackage{tikz, tikz-3dplot, tikzpagenodes}
\usepackage{graphicx, xcolor, geometry, mdframed}
\usepackage{enumitem, nicematrix}
\usepackage{mathtools, mathrsfs}
\usepackage{caption, subcaption}
\usepackage{bookmark, xifthen}
\usepackage{transparent}

\usetikzlibrary{matrix, positioning, patterns, decorations.markings, arrows, arrows.meta, backgrounds, math, cd}
\tikzset{->-/.style={decoration={ markings, mark=at position #1 with {\arrow{>}}},postaction={decorate}}}

\definecolor{darkBlue}{RGB}{0, 0, 138}
\hypersetup{colorlinks=true, allcolors=darkBlue}
\theoremstyle{definition}\newtheorem{question}{\color{red}Question}

\input{setup/macros.sty}

\begin{document}
    \title{Tree-like Graphings}
    \author{Zhaoshen Zhai}
    \date{\today}

    \begin{abstract}
        We present a streamlined exposition of a construction presented in \cite{CPTT23}, where it is proven that every locally-finite Borel graph with each component a quasi-tree induces a canonical treeable equivalence relation.
    \end{abstract}

    \maketitle

    \section{Introduction}

    The purpose of this note is to provide a streamlined proof of a particular case of a construction presented in \cite{CPTT23}, in order to better understand the general formalism developed therein. We attempt to make this note self-contained, but nevertheless urge the reader to refer to the original paper for more detailed discussion and generalizations of the results we have selected to include here.

    \subsection{Treeings of Equivalence Relations}

    A \textit{countable Borel equivalence relation (CBER)} on a standard Borel space $X$ is a Borel equivalence relation $E\subseteq X^2$, such that each class is countable. We are interested in special types of \textit{graphings} on a CBER $E\subseteq X^2$, i.e. a Borel graph $G\subseteq X^2$ whose connectedness relation is precisely $E$. For instance, a graphing of $E$ such that each component is a tree is called a \textit{treeing} of $E$, and the CBERs that admit a treeing are said to be \textit{treeable}. The main results of \cite{CPTT23} is to provide new sufficient criteria for treeability of CBERs, and in particular, they prove the following

    \begin{mainTheorem}[\cite{CPTT23}*{Theorem 1.1}, Corollary \ref{cor:quasi-treeing_implies_treeing}]
        If a CBER admits a locally-finite graphing whose components are quasi-trees, then it is treeable.
    \end{mainTheorem}

    Recall that two metric spaces $X$ and $Y$ are \textit{quasi-isometric} if they are isometric up to a bounded multiplicative and additive error, and $X$ is a \textit{quasi-tree} if it is quasi-isometric to a simplicial tree; see \cite{DK18}.

    \subsection{Outline of the Proof}

    Roughly speaking, the existence of a quasi-isometry $G|C\to T_C$ to a simplicial tree $T_C$ for each component $C\subseteq G$ induces a collection $\mc{H}(C)$ of `cuts' (subsets $H\subseteq C$ with finite boundary such that both $H$ and $C\comp H$ are connected), which are `tree-like' in the sense that
    \begin{enumerate}
        \item[1.] $\mc{H}(C)$ is a \textit{walling}: each vertex $x\in C$ is on the boundary of finitely-many $H\in\mc{H}(C)$, and
        \item[2.] $\mc{H}(C)$ is \textit{dense towards ends}: each end in $G|C$ has a neighborhood basis in $\mc{H}(C)$.
    \end{enumerate}
    These cuts provide exactly the data to construct a `median graph' whose vertices are `ultrafilters' thereof, and condition (2) ensures that this graph has finite `hyperplanes'. This finiteness condition allows us to apply a Borel `cycle-cutting' algorithm to obtain a subtree thereof. Each step above can be done in a uniform way to each component $C\subseteq G$, giving us the desired treeing of the CBER.

    

    \section{Main Constructions}

    \begin{corollary}\label{cor:quasi-treeing_implies_treeing}
        If a CBER admits a locally-finite graphing whose components are quasi-trees, then it is treeable.
    \end{corollary}

    \subsection{Wallings of Cuts}

    \subsection{Median Graph of Orientations}

    \subsection{Canonical Construction of a Sub-treeing}

    \begin{bibdiv}
        \begin{biblist}*{labels={alphabetic}}
            \bibselect{setup/bibliography}
        \end{biblist}
    \end{bibdiv}
\end{document}
