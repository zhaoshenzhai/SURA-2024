\documentclass{beamer}

\usepackage{amsfonts, amsmath, amssymb, amsthm, amsrefs}
\usepackage{tikz, tikz-3dplot, tikzpagenodes}
\usepackage{graphicx, xcolor, geometry, mdframed}
\usepackage{enumitem, nicematrix}
\usepackage{mathtools, mathrsfs}
\usepackage{caption, subcaption}
\usepackage{bookmark, xifthen}
\usepackage{transparent}

\usetikzlibrary{matrix, positioning, patterns, decorations.markings, arrows, arrows.meta, backgrounds, math, cd}
\tikzset{->-/.style={decoration={ markings, mark=at position #1 with {\arrow{>}}},postaction={decorate}}}

\definecolor{darkBlue}{RGB}{0, 0, 138}
\hypersetup{colorlinks=true, allcolors=darkBlue}
\theoremstyle{definition}\newtheorem{question}{\color{red}Question}

\input{setup/macros.sty}

\title{\normalsize{Tree-like graphings of countable Borel equivalence relations}}
\subtitle{\scriptsize{An exposition to\\\textit{Tree-like graphings, wallings, and median graphings of equivalence relations}\\by Ruiyuan Chen, Antoine Poulin, Ran Tao, and Anush Tserunyan}}
\author{\small Zhaoshen Zhai}
\date{\footnotesize October 1, 2024}

\usecolortheme{seahorse}
\usefonttheme{serif}
\useinnertheme{rounded}

\begin{document}\frame{\titlepage}
    \begin{frame}{Countable Borel Equivalence Relations}
        \begin{definition}
            A \textit{countable Borel equivalence relation (CBER)} on a standard Borel space $X$ is a Borel equivalence relation $E\subseteq X^2$ such that each $E$-class is countable.
        \end{definition}

        \pause
        \vspace{-0.1in}

        \begin{example}
            Any Borel action $\Gamma\act X$ of a countable (discrete) group on a standard Borel space induces its \textit{orbit equivalence relation} $E^X_\Gamma$, which is a CBER.
        \end{example}

        \pause
        \vspace{-0.15in}

        \begin{figure}[h]
            \center
            \includegraphics[width=0.9\textwidth]{img/group_action.png}
        \end{figure}
    \end{frame}
    \begin{frame}{Smooth and Hyperfinite CBERs}
        \vspace{-0.1in}
        \begin{example}[Smooth]
            \begin{itemize}
                \item[\scriptsize$\blob$] Identity relation $=$ on a standard Borel space, say $\R$ or $2^\N$.
                    \pause
                \item[\scriptsize$\blob$] $\Z$-coset equivalence on $\R$: $xE^\R_\Z y$ iff $x-y\in\Z$.
            \end{itemize}
        \end{example}

        \pause
        \vspace{-0.1in}

        \begin{example}[Hyperfinite]
            $E_0$ on $2^\N$, where $xE_0y$ iff $\ex n\in\N,\fa m\geq n:x_m=y_m$.
        \end{example}

        \pause

        This CBER is \textit{hyperfinite}: $E_0=\bigcup_nF_n$ for an increasing sequence $F_0\subseteq F_1\cdots$ of finite Borel equivalence relations:
        \pause
        \vspace{-0.1in}
        \begin{equation*}
            xF_ny\ \ \ \ \leftrightarrow\ \ \ \ \fa m\geq n:x_m=y_m.
        \end{equation*}

        \pause
        \vspace{-0.15in}

        \begin{theorem}[Slaman-Steel, Weiss]
            Let $E$ be a CBER on a standard Borel space $X$. TFAE:
            \begin{itemize}
                \item[$1.$] $E$ is hyperfinite. {\color{gray}\footnotesize $E=\bigcup_nF_n$ where $F_0\subseteq F_1\subseteq\cdots$ are FBERs.}
                \item[$2.$] $E$ is induced by a Borel $\Z$-action. {\color{gray}\footnotesize $E=E_\Z^X$ for some $\Z\act X$.}
            \end{itemize}
        \end{theorem}
    \end{frame}
    \begin{frame}{Smooth and Hyperfinite CBERs}
        \vspace{-0.34in}
        \begin{example}[Smooth]
            \begin{itemize}
                \item[\scriptsize$\blob$] Identity relation $=$ on a standard Borel space, say $\R$ or $2^\N$.
                \item[\scriptsize$\blob$] $\Z$-coset equivalence on $\R$: $xE^\R_\Z y$ iff $x-y\in\Z$.
            \end{itemize}
        \end{example}

        \vspace{-0.1in}

        \begin{example}[Hyperfinite]
            $E_0$ on $2^\N$, where $xE_0y$ iff $\ex n\in\N,\fa m\geq n:x_m=y_m$.
        \end{example}

        \vspace{-0.15in}

        \begin{figure}[h]
            \center
            \includegraphics[width=0.9\textwidth]{img/smooth_hyperfinite.png}
        \end{figure}
    \end{frame}
    \begin{frame}{Graphing of a CBER}
        \begin{definition}
            A \textit{graphing} of a CBER $E$ on $X$ is a Borel graph $G\subseteq X^2$ whose connected relation is $E$, i.e., $xEy\leftrightarrow xG\cdots Gy$ for all $x,y\in X$.
        \end{definition}

        \pause
        \vspace{-0.15in}

        \begin{figure}[h]
            \center
            \includegraphics[width=0.7\textwidth]{img/group_action.png}
        \end{figure}
        \vspace{-0.2in}
        \begin{figure}[h]
            \center
            \includegraphics[width=0.7\textwidth]{img/smooth_hyperfinite.png}
        \end{figure}
    \end{frame}
    \begin{frame}{Treeings and Treeability}
        \begin{definition}
            A \textit{treeing} of a CBER $E$ is an acyclic graphing, and a CBER $E$ is said to be \textit{treeable} if it admits a treeing.
        \end{definition}

        \vspace{-0.14in}

        \begin{figure}[h]
            \center
            \includegraphics[width=0.7\textwidth]{img/not_treeable.png}
        \end{figure}
        \vspace{-0.2in}
        \begin{figure}[h]
            \center
            \includegraphics[width=0.7\textwidth]{img/smooth_hyperfinite.png}
        \end{figure}
    \end{frame}
    \begin{frame}{Treeable CBERs}
        \vspace{-0.2in}

        \begin{example}[Free Actions]
            Free actions of a free group $F_r\act X$.
        \end{example}

        \begin{figure}[h!]
            \vspace{-0.8in}
            \hfill
            \includegraphics[width=0.3\textwidth]{img/free_group_action.png}
        \end{figure}

        \pause
        \vspace{-0.28in}

        \begin{theorem}[JKL02]
            Free actions of virtually-free groups are treeable.
        \end{theorem}

        \pause

        \begin{theorem}[GdlH90]
            Every finitely-generated group whose Cayley graph is a quasi-tree is virtually-free, and hence treeable.
        \end{theorem}

        \pause

        \begin{question}[Robin Tucker-Drob; 2015]
            Is the class of treeable CBERs robust under quasi-isometries?
        \end{question}
    \end{frame}
    \begin{frame}{Main Result}
        \begin{theorem}[Chen, Poulin, Tao, Tserunyan; 2023+]
            If a CBER $E$ admits a locally-finite graphing such that each component is a quasi-tree, then $E$ is treeable.
        \end{theorem}

        \pause
        \vspace{0.1in}

        \scriptsize{
            Two metric spaces $X,Y$ are \textit{quasi-isometric} if they are isometric up to a bounded multiplicative and additive error; $X$ is a \textit{quasi-tree} if it is quasi-isometric to a tree.
            \vspace{-0.3in}
            \begin{figure}[h]
                \center
                \includegraphics[width=0.9\textwidth]{img/quasi_tree.png}
            \end{figure}
        }
    \end{frame}
    \begin{frame}{Game Plan}
        \begin{center}
            \scalebox{0.85}{
                \hspace*{-0.62in}
                \def\hsp{3.75}
                \def\vsp{1.3}
                \def\vsh{0.05}
                \begin{tikzpicture}
                    \draw[dashed, rounded corners, opacity=0.2] (-1.7,0.5) -- (-1.7,-0.4-\vsp) -- (0.75,-0.4-\vsp) -- (3,-0.4) -- (3*\hsp+1.5,-0.4) -- (3*\hsp+1.5,0.5) -- cycle;
                    \draw[dashed, rounded corners, opacity=0.2] (2.5,0.55-\vsp) rectangle (3*\hsp+1.5,-0.45-\vsp);

                    \draw (0,\vsh) circle (0in) node{Quasi-treeing};
                    \draw (3*\hsp,\vsh) circle (0in) node{Treeing};

                    \pause

                    \only<2>{\node[inner sep=0pt] at (5,-5) {\includegraphics[width=0.9\textwidth]{img/plan/0_base.png}};}
                    \only<3>{{\transparent{0.5}\node[inner sep=0pt] at (5,-5) {\includegraphics[width=0.9\textwidth]{img/plan/0_base.png}};}}
                    \onslide<4-8>{{\transparent{0.1}\node[inner sep=0pt] at (5,-5) {\includegraphics[width=0.9\textwidth]{img/plan/0_base.png}};}}

                    \draw[->] (0,-0.2) -- (0,-\vsp+0.3);
                    \draw (0,\vsh-\vsp) circle (0in) node{Quasi-tree};

                    \pause

                    \draw (\hsp,-\vsp) circle (0in) node{$\begin{gathered} \textrm{Finitely-sep.}\\[-4pt] \textrm{family of cuts} \end{gathered}$};
                    \draw[->] (1,\vsh-\vsp) -- (\hsp-1.1,\vsh-\vsp);

                    \only<3>{\node[inner sep=0pt] at (4.657,-5.34) {\includegraphics[width=0.9\textwidth]{img/plan/1_cut.png}};}

                    \pause

                    \onslide<4,7>{\node[inner sep=0pt] at (5.8,-5.35) {\includegraphics[width=0.9\textwidth]{img/plan/2_cuts.png}};}
                    \only<5>{{\transparent{0.6}\node[inner sep=0pt] at (5.8,-5.35) {\includegraphics[width=0.9\textwidth]{img/plan/2_cuts.png}};}}
                    \onslide<6,8>{{\transparent{0.3}\node[inner sep=0pt] at (5.8,-5.35) {\includegraphics[width=0.9\textwidth]{img/plan/2_cuts.png}};}}

                    \pause

                    \draw (2*\hsp,-\vsp) circle (0in) node{$\begin{gathered} \textrm{Median graph w/}\\[-4pt] \textrm{finite hyperplanes} \end{gathered}$};
                    \draw[->] (\hsp+1.1,\vsh-\vsp) -- (2*\hsp-1.5,\vsh-\vsp);

                    \only<5>{\node[inner sep=0pt] at (5.5,-5) {\includegraphics[width=0.83\textwidth]{img/plan/3_median.png}};}
                    \onslide<6-8>{{\transparent{0.4}\node[inner sep=0pt] at (5.5,-5) {\includegraphics[width=0.83\textwidth]{img/plan/3_median.png}};}}

                    \pause

                    \draw (3*\hsp,-\vsp) circle (0in) node{$\begin{gathered} \textrm{`Canonical'}\\[-4pt] \textrm{spanning tree} \end{gathered}$};
                    \draw[->] (2*\hsp+1.5,\vsh-\vsp) -- (3*\hsp-1.1,\vsh-\vsp);

                    \onslide<6,8>{\node[inner sep=0pt] at (5.76,-4.65) {\includegraphics[width=0.83\textwidth]{img/plan/4_tree.png}};}
                    \only<7>{{\transparent{0.2}{\node[inner sep=0pt] at (5.76,-4.65) {\includegraphics[width=0.83\textwidth]{img/plan/4_tree.png}};}}}

                    \pause

                    \draw (\hsp,0) circle (0in) node{$\begin{gathered} \textrm{Dense family}\\[-4pt] \textrm{of cuts} \end{gathered}$};
                    \draw[->] (1,-1) -- (\hsp-1.1,0);

                    \node[inner sep=0pt] at (5.1,-5.2) {\includegraphics[width=0.9\textwidth]{img/plan/5_ends.png}};

                    \pause

                    \draw[->] (\hsp+1.1,\vsh) -- (3*\hsp-1.1,\vsh);
                    \draw[->] (3*\hsp,0.4-\vsp) -- (3*\hsp,-0.2);
                    \draw (\hsp+1.1,-1) to [out=35, in=180] (2*\hsp,\vsh);
                \end{tikzpicture}
            }
        \end{center}
    \end{frame}
    \begin{frame}{The End}
        \begin{center}
            Thank you!
        \end{center}
    \end{frame}
\end{document}
