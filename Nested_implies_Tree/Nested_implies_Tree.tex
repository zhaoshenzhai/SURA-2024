\documentclass{amsart}
\usepackage[hidelinks]{hyperref}                                            % links
\usepackage{pgfplots}\pgfplotsset{compat=1.18}                              % plots
\usepackage{amsfonts, amsmath, amssymb, amsthm}                             % basic maths commands
\usepackage{mathtools, mathrsfs}                                            % more maths commands
\usepackage{graphicx}                                                       % images and other graphics
\usepackage{geometry}                                                       % page layout
\usepackage{tikz, tikz-3dplot, tikzpagenodes}                               % maths figures
\usepackage{caption, subcaption}                                            % captions outside float
\usepackage{xcolor}                                                         % more colors
\usepackage{enumitem}                                                       % enumerate and itemize indents
\usepackage{nicematrix}                                                     % matrices and tables

\usetikzlibrary{matrix, positioning, patterns, decorations.markings, arrows, arrows.meta, backgrounds, math, cd}

\hypersetup{colorlinks=true, allcolors=magenta}
\definecolor{darkGreen}{HTML}{00A000}
\newgeometry{margin = 1in}

\newtheorem{theorem}{Theorem}[section]
\newtheorem{mainTheorem}{Theorem}
\newtheorem{proposition}[theorem]{Proposition}
\newtheorem{lemma}[theorem]{Lemma}
\newtheorem{corollary}[theorem]{Corollary}
\theoremstyle{definition}\newtheorem{example}[theorem]{Example}
\theoremstyle{definition}\newtheorem{definition}[theorem]{Definition}
\theoremstyle{definition}\newtheorem{remark}[theorem]{Remark}
\theoremstyle{definition}\newtheorem{notation}[theorem]{Notation}
\renewcommand{\themainTheorem}{\Alph{mainTheorem}}
\newtheorem*{theorem*}{Theorem}
\newtheorem*{proposition*}{Proposition}
\newtheorem*{lemma*}{Lemma}
\newtheorem*{corollary*}{Corollary}
\theoremstyle{definition}\newtheorem*{example*}{Example}
\theoremstyle{definition}\newtheorem*{definition*}{Definition}
\theoremstyle{definition}\newtheorem*{remark*}{Remark}
\theoremstyle{definition}\newtheorem*{notation*}{Notation}

% Operators
    \newcommand{\id}{\operatorname{id}}
    \newcommand{\im}{\operatorname{im}}
    \newcommand{\rk}{\operatorname{rk}}
    \newcommand{\ch}{\operatorname{ch}}
    \newcommand{\tr}{\operatorname{tr}}
    \newcommand{\tp}{\operatorname{tp}}
    \newcommand{\qd}{\operatorname{qd}}
    \newcommand{\ON}{\operatorname{ON}}
    \newcommand{\GL}{\operatorname{GL}}
    \newcommand{\SL}{\operatorname{SL}}
    \newcommand{\Id}{\operatorname{Id}}
    \newcommand{\Th}{\operatorname{Th}}
    \newcommand{\Cn}{\operatorname{Cn}}
    \newcommand{\Bl}{\operatorname{Bl}}
    \newcommand{\Cl}{\operatorname{Cl}}
    \newcommand{\LT}{\operatorname{LT}}
    \newcommand{\dom}{\operatorname{dom}}
    \newcommand{\ran}{\operatorname{ran}}
    \newcommand{\cdm}{\operatorname{cdm}}
    \newcommand{\sgn}{\operatorname{sgn}}
    \newcommand{\lcm}{\operatorname{lcm}}
    \newcommand{\ord}{\operatorname{ord}}
    \newcommand{\cvx}{\operatorname{cvx}}
    \newcommand{\Aut}{\operatorname{Aut}}
    \newcommand{\Inn}{\operatorname{Inn}}
    \newcommand{\Out}{\operatorname{Out}}
    \newcommand{\End}{\operatorname{End}}
    \newcommand{\Mat}{\operatorname{Mat}}
    \newcommand{\Obj}{\operatorname{Obj}}
    \newcommand{\Hom}{\operatorname{Hom}}
    \newcommand{\Tor}{\operatorname{Tor}}
    \newcommand{\Ann}{\operatorname{Ann}}
    \newcommand{\Sym}{\operatorname{Sym}}
    \newcommand{\Cov}{\operatorname{Cov}}
    \newcommand{\Orb}{\operatorname{Orb}}
    \newcommand{\Sat}{\operatorname{Sat}}
    \newcommand{\Thm}{\operatorname{Thm}}
    \newcommand{\Der}{\operatorname{Der}}
    \newcommand{\Age}{\operatorname{Age}}
    \newcommand{\Div}{\operatorname{Div}}
    \newcommand{\PGL}{\operatorname{PGL}}
    \newcommand{\rank}{\operatorname{rank}}
    \newcommand{\proj}{\operatorname{proj}}
    \newcommand{\diag}{\operatorname{diag}}
    \newcommand{\eval}{\operatorname{eval}}
    \newcommand{\cont}{\operatorname{cont}}
    \newcommand{\diam}{\operatorname{diam}}
    \newcommand{\mult}{\operatorname{mult}}
    \newcommand{\Core}{\operatorname{Core}}
    \newcommand{\Term}{\operatorname{Term}}
    \newcommand{\Taut}{\operatorname{Taut}}
    \newcommand{\Sent}{\operatorname{Sent}}
    \newcommand{\Skew}{\operatorname{Skew}}
    \newcommand{\Frac}{\operatorname{Frac}}
    \newcommand{\Stab}{\operatorname{Stab}}
    \newcommand{\Isom}{\operatorname{Isom}}
    \newcommand{\Meas}{\operatorname{Meas}}
    \newcommand{\Diag}{\operatorname{Diag}}
    \newcommand{\Sing}{\operatorname{Sing}}
    \newcommand{\coker}{\operatorname{coker}}
    \newcommand{\preim}{\operatorname{preim}}
    \newcommand{\Graph}{\operatorname{Graph}}
    \newcommand{\UnSat}{\operatorname{UnSat}}
    \newcommand{\Axioms}{\operatorname{Axioms}}
    \renewcommand{\Re}{\operatorname{Re}}
    \renewcommand{\Im}{\operatorname{Im}}
    \renewcommand{\div}{\operatorname{div}}
    \renewcommand{\span}{\operatorname{span}}
    \renewcommand{\Form}{\operatorname{Form}}

% Math notations
    % Set Theory, Category Theory, and Logic
        \newcommand{\fa}{\forall}
        \newcommand{\ex}{\exists}
        \newcommand{\MP}{\textrm{MP}}
        \newcommand{\PA}{\textrm{PA}}
        \newcommand{\PL}{{\rm P{\small L}}}
        \newcommand{\DLO}{\textrm{DLO}}
        \newcommand{\ZFC}{\textrm{ZFC}}
        \newcommand{\ACF}{\textrm{ACF}}
        \newcommand{\FOL}{{\rm F{\small OL}}}
        \newcommand{\iso}{\cong}
        \newcommand{\pow}{\mathcal{P}}
        \newcommand{\comp}{\setminus}
        \newcommand{\into}{\hookrightarrow}
        \newcommand{\onto}{\twoheadrightarrow}
        \newcommand{\parto}{\rightharpoonup}
        \newcommand{\eqnum}{\approx}
        \newcommand{\natiso}{\simeq}
        \newcommand{\proves}{\vdash}
        \newcommand{\adjoin}{^\smallfrown}
        \newcommand{\nproves}{\nvdash}
        \newcommand{\symdiff}{\vartriangle}
        \newcommand{\infrule}{\rightsquigarrow}
        \newcommand{\eleminto}{\into_e}
        \newcommand{\elemequiv}{\equiv}
        \newcommand{\substruct}{<}
        \newcommand{\supstruct}{>}
        \newcommand{\elemembed}{\preceq}
        \newcommand{\elemextend}{\succeq}
        \newcommand{\substructeq}{\leq}
        \newcommand{\supstructeq}{\geq}
        \renewcommand{\em}{\varnothing}
        \renewcommand{\vec}[1]{\bar{#1}}

    % Complexity Theory
        \newcommand{\NP}{\mathsf{NP}}
        \newcommand{\coNP}{\mathsf{coNP}}

    % Categories
        \newcommand{\cat}[1]{\textbf{#1}}
        \newcommand{\catset}{\cat{Set}}
        \newcommand{\catgrp}{\cat{Grp}}
        \newcommand{\catmon}{\cat{Mon}}
        \newcommand{\cattop}{\cat{Top}}
        \newcommand{\catmet}{\cat{Met}}
        \newcommand{\catrel}{\cat{Rel}}
        \newcommand{\catord}{\cat{Ord}}
        \newcommand{\catscat}{\cat{Cat}}
        \newcommand{\catlscat}{\cat{CAT}}
        \newcommand{\catgrpd}{\cat{Grpd}}
        \newcommand{\catring}{\cat{Ring}}
        \newcommand{\cathtop}{\cat{hTop}}
        \newcommand{\catptop}{\cat{Top}_\blob}
        \newcommand{\catphtop}{\cat{hTop}_\blob}
        \newcommand{\catabgrp}{\cat{Ab}}
        \newcommand{\catman}[1][\infty]{\cat{Man}_{#1}}
        \newcommand{\cathom}[1][\mc{L}]{#1\textrm{-}\cat{Hom}}
        \newcommand{\catemb}[1][\mc{L}]{#1\textrm{-}\cat{Emb}}
        \newcommand{\catmod}[1][R]{\prescript{}{#1}{\cat{Mod}}}
        \newcommand{\catrmod}[1][R]{\cat{Mod}_{#1}}
        \newcommand{\catcov}[1][X]{\cat{Cov}\l(#1\r)}
        \newcommand{\catfgmod}[1][R]{\cat{fg}_{#1}\cat{Mod}}
        \newcommand{\catvect}[1][k]{\prescript{}{#1}{\cat{Vect}}}
        \newcommand{\catfgvect}[1][k]{\cat{fg}_{#1}\cat{Vect}}
        \newcommand{\catalg}[1][R]{\prescript{}{#1}{\cat{Alg}}}
        \newcommand{\catgset}[1]{\prescript{}{#1}{\cat{Set}}}
        \newcommand{\catmodel}[2][\mc{L}]{\mc{M}_#1\!\l(#2\r)}
        \newcommand{\catrep}[2][\,]{\cat{Rep}_{#1\!}\l(#2\r)}
        \newcommand{\catfgrep}[2][\,]{\cat{fgRep}_{#1\!}\l(#2\r)}

    % Analysis
        \newcommand{\BV}{BV}
        \newcommand{\del}{\partial}
        \newcommand{\incto}{\nearrow}
        \newcommand{\decto}{\searrow}
        \newcommand{\abscont}{\ll}
        \newcommand{\esssup}{\operatorname{ess-sup}}
        \renewcommand{\d}{\mathrm{d}}

    % Topology
        \newcommand{\rel}{\,\operatorname{rel}\,}
        \newcommand{\tcl}{\operatorname{cl}}
        \newcommand{\scl}{\operatorname{scl}}
        \newcommand{\tint}{\operatorname{int}}
        \newcommand{\sint}{\operatorname{sint}}
        \newcommand{\htopeq}{\simeq}
        \newcommand{\pathto}{\rightsquigarrow}

    % Linear Algebra
        \newcommand{\dual}{\wedge}
        \newcommand{\adj}{\ast}
        \newcommand{\trans}{\mathsf{T}}
        \newcommand{\inprod}[2]{\l\langle{#1},{#2}\r\rangle}

    % Group Theory
        \newcommand{\act}{\curvearrowright}
        \newcommand{\semi}{\rtimes}
        \newcommand{\nsubgrp}{\triangleleft}
        \newcommand{\nsupgrp}{\triangleright}
        \newcommand{\nsubgrpeq}{\trianglelefteq}
        \newcommand{\nsupgrpeq}{\trianglerighteq}

    % Number Theory
        \newcommand{\divides}{\,|\,}
        \newcommand{\ndivides}{\nmid}
        \renewcommand{\mod}[1]{\l(\operatorname{mod}\,#1\r)}

    % Algebraic Geometry
        \newcommand{\ratto}{\dashrightarrow}

    % Misc
        \newcommand{\st}{:}
        \newcommand{\tpl}[1]{\l(#1\r)}
        \newcommand{\gen}[1]{\l\langle#1\r\rangle}
        \renewcommand{\bar}{\overline}

% Math others
    % Number Systems
        \newcommand{\N}{\mathbb{N}}
        \newcommand{\Z}{\mathbb{Z}}
        \newcommand{\Q}{\mathbb{Q}}
        \newcommand{\R}{\mathbb{R}}
        \newcommand{\C}{\mathbb{C}}
        \newcommand{\F}{\mathbb{F}}
        \newcommand{\E}{\mathbb{E}}
        \newcommand{\A}{\mathbb{A}}
        \renewcommand{\S}{\mathbb{S}}
        \renewcommand{\P}{\mathbb{P}}
        \renewcommand{\H}{\mathbb{H}}

% LaTeX/MathJax
    % Fonts
        \newcommand{\mc}[1]{\mathcal{#1}}
        \newcommand{\ms}[1]{\mathscr{#1}}
        \newcommand{\mb}[1]{\mathbb{#1}}
        \newcommand{\mf}[1]{\mathfrak{#1}}
        \renewcommand{\it}[1]{\textit{#1}}
        \renewcommand{\bf}[1]{\textbf{#1}}
        \renewcommand{\sf}[1]{\textsf{#1}}
        \renewcommand{\phi}{\varphi}
        \renewcommand{\epsilon}{\varepsilon}

    % Meta
        \newcommand{\blob}{\bullet}
        \newcommand{\slot}{-}
        \newcommand{\cref}[1]{\tag{$\,#1\,$}}
        \newcommand{\qedin}{\tag*{$\blacksquare$}}
        \newcommand{\exqedin}{\tag*{$\blacklozenge$}}
        \renewcommand{\l}{\left}
        \renewcommand{\r}{\right}
        \renewcommand{\qed}{\phantom\qedhere\hfill$\blacksquare$}
        \renewcommand{\ref}[1]{\l(\,#1\,\r)}


\begin{document}
    \title{Tree of Orientations on a Nested Collection of Cuts}
    \author{Zhaoshen Zhai}
    \maketitle

    \section{Introduction}
    Let $\mc{C}\subseteq2^X$ be a collection of non-empty subsets of a set $X$. With the definitions in Section \ref{prelim}, we prove the following
    \begin{theorem}\label{main}
        If $\mc{C}$ is nested, then the graph $\mc{T}_\mc{C}$, whose:
        \begin{itemize}
            \item Vertices are finitely-based orientations on $\mc{C}$; and whose
            \item Neighbors of $\mc{U}\in V(\mc{T}_\mc{C})$ are $\mc{U}\symdiff\l\{A,A^c\r\}$ for every minimal $A\in\mc{U}$ with $A^c\in\mc{C}$;
        \end{itemize}
        is acyclic. Furthermore, $\mc{T}_\mc{C}$ is a tree iff $\mc{C}$ is closed under complements.
    \end{theorem}

    In particular, this applies to when $(X,G)$ is a graph and $\mc{C}$ is a nested collection of cuts on $X$.

    \section{Preliminaries}\label{prelim}

    \subsection{Orientations}

    Since we do not assume that $\mc{C}$ is closed under complements, we slightly modify the definition of orientations, as follows.

    \begin{definition}
        An \textit{orientation} on $\mc{C}$ is a subset $\mc{U}\subseteq\mc{C}$ such that
        \begin{enumerate}
            \item[1.] \textit{(Upward-closure)}. If $A\in\mc{U}$ and $B\in\mc{C}$ contains $A$, then $B\in\mc{U}$.
            \item[2.] \textit{(Ultra)}. If $A,A^c\in\mc{C}$, then either $A\in\mc{U}$ or $A^c\in\mc{U}$, but not both.
        \end{enumerate}
    \end{definition}

    \begin{remark}
        This coincides with the standard definition when $\mc{C}$ is a subpocset of $2^X$.
    \end{remark}

    \begin{lemma}\label{tree-well-defined}
        If $\mc{U}\subseteq\mc{C}$ is an orientation, then for any $\subseteq$-minimal $A\in\mc{U}$ with $A^c\in\mc{C}$, so is $\mc{U}\symdiff\l\{A,A^c\r\}$.
    \end{lemma}
    \begin{proof}
        That $\mc{V}\coloneqq\mc{U}\symdiff\l\{A,A^c\r\}=\mc{U}\cup\l\{A^c\r\}\comp\l\{A\r\}$ is upward-closed follows from $\subseteq$-minimality of $A$. Now, if $B,B^c\in\mc{C}$ and $B^c\not\in\mc{V}$, we need to show that $B\in\mc{V}$.

        To this end, note that $B^c\not\in\mc{V}$ implies $A\neq B$ and either $B=A^c$ or $B^c\not\in\mc{U}$. The former case follows from $A^c\in\mc{V}$, and for the latter, we have $B\in\mc{U}\comp\l\{A\r\}$ since $\mc{U}$ is ultra.
    \end{proof}

    \begin{remark}\label{tree-no-loops}
        In the above notations, clearly $\mc{U}\neq\mc{U}\symdiff\l\{A,A^c\r\}$. Furthermore, for any other such orientation $\mc{U}'$ and $A'\in\mc{U}'$, that $\mc{U}=\mc{U}'$ and $\mc{U}\symdiff\l\{A,A^c\r\}=\mc{U}'\symdiff\l\{A',A'^c\r\}$ together imply $A=A'$.
    \end{remark}

    \begin{definition}
        A \textit{base} for an orientation $\mc{U}\subseteq\mc{C}$ is a $\subseteq$-minimal subset $\mc{B}\subseteq\mc{U}$ such that $\mc{U}=\,\,\uparrow\!\mc{B}$, where
        \begin{equation*}
            \uparrow\!\mc{B}\coloneqq\bigcup_{B\in\mc{B}}\uparrow\!B\coloneqq\bigcup_{B\in\mc{B}}\l\{A\in\mc{C}\st A\supseteq B\r\}.
        \end{equation*}
    \end{definition}

    \begin{definition}
        A collection $\mc{C}$ is said to be \textit{nested} if every $C_1,C_2\in\mc{C}$ has an empty corner, i.e., $C_1^i\cap C_2^j=\em$ for some $i,j\in\l\{1,-1\r\}$, where $C^i\coloneqq C$ if $i=1$ and $C^i\coloneqq C^c$ if $i=-1$.
    \end{definition}

    \begin{remark}
        If $\mc{C}$ is nested, then every $\subseteq$-minimal $B\in\mc{C}$ induces an orientation $\uparrow\!B\coloneqq\l\{A\in\mc{C}\st A\supseteq B\r\}$, called the \textit{principal} orientation. Indeed, $\uparrow\!B$ is clearly upward-closed, and if $A,A^c\in\mc{C}$, then, by $\subseteq$-minimality of $B$ and nestedness of $\mc{C}$, either $A\supseteq B$ or $A^c\supseteq B$ (but clearly not both).

        This construction generalizes to any collection $\mc{B}\subseteq\mc{C}$ with each $B\in\mc{B}$ being $\subseteq$-minimal, in that $\uparrow\!\mc{B}$ is an orientation on $\mc{C}$.
    \end{remark}

    \begin{definition}
        An orientation $\mc{U}\subseteq\mc{C}$ is said to be \textit{finitely-based} if it admits a finite base.
    \end{definition}

    \begin{remark}\label{tree-finitely-based}
        If $\mc{U}=\,\,\uparrow\!\l\{B_1,\dots,B_n\r\}$ is finitely-based, then for any $\subseteq$-minimal $A\in\mc{U}$ with $A^c\in\mc{C}$, so is the orientation $\mc{V}\coloneqq\mc{U}\symdiff\l\{A,A^c\r\}$. Indeed, $A=B_i$ for some $1\leq i\leq n$, and $\mc{V}=\,\,\uparrow\!(\l\{A\r\}\cup\l\{B_j\r\}_{j\neq i})$.
    \end{remark}

    \section{The graph $\mc{T}_\mc{C}$}

    Fix a nested collection of non-empty subsets of a set $X$. Using Lemma \ref{tree-well-defined} and Remarks \ref{tree-no-loops} and \ref{tree-finitely-based}, we construct a graph $\mc{T}_\mc{C}$ whose:
    \begin{itemize}
        \item \textit{Vertices} of $\mc{T}_\mc{C}$ are finitely-based orientations on $\mc{C}$.
        \item \textit{Neighbors} of $\mc{U}\in V(\mc{T}_\mc{C})$ are the finitely-based orientations $\mc{U}\symdiff\l\{A,A^c\r\}$ for every minimal $A\in\mc{U}$.
    \end{itemize}

    The goal of this section is to establish Theorem \ref{main}, stating that $\mc{T}_\mc{C}$ is acyclic (Proposition \ref{acyclic}), and furthermore, $\mc{T}_\mc{C}$ is a tree precisely when $\mc{C}$ is closed under complements (Proposition \ref{tree}).

    \subsection{Paths in $\mc{T}_\mc{C}$}

    To show that $\mc{T}_\mc{C}$ is acyclic, we characterize backtracking paths in $\mc{T}_\mc{C}$ as follows.

    \begin{definition}
        Fix $\mc{U}_0\in V(\mc{T}_\mc{C})$ and $\alpha\leq\omega$. A sequence $(A_n)_{n<\alpha}\subseteq\mc{C}$ is said to \textit{represent a path from $\mc{U}_0$} if $(\mc{U}_n)_{n\leq\alpha}$, defined by $\mc{U}_n\coloneqq\mc{U}_{n-1}\symdiff\l\{A_{n-1}\r\}$ for every $1\leq n\leq\alpha$, is a path in $\mc{T}_\mc{C}$ with each $A_n\in\mc{U}_n$.
    \end{definition}

    \begin{remark}
        Any path in $\mc{T}_\mc{C}$ is represented by its sequence of flipped basis elements.
    \end{remark}

    \begin{lemma}\label{no-backtrack}
        Let $\alpha\geq3$. A path in $\mc{T}_\mc{C}$ from $\mc{U}_0$ represented by $(A_n)_{n\leq\alpha}$ has no backtracking iff $A_n\neq A_{n-1}^c$ for every $1\leq n<\alpha$.
    \end{lemma}
    \begin{proof}
        Take $2\leq n\leq\alpha$. It suffices to show that $\mc{U}_{n-2}=\mc{U}_n$ iff $A_{n-1}=A_{n-2}^c$.
        \begin{itemize}
            \item[($\Rightarrow$).] We have by definition that $\mc{U}_n=\mc{U}_{n-2}\cup\l\{A_{n-1}^c,A_{n-2}^c\r\}\comp\l\{A_{n-1},A_{n-2}\r\}$, so since $A_{n-2}\in\mc{U}_{n-2}=\mc{U}_n$, we have $A_{n-2}=A_{n-1}^c$ as desired.
            \item[($\Leftarrow$).] Again by definition, by noting that the basis-flipping cancels out.\qed
        \end{itemize}
    \end{proof}

    \begin{lemma}\label{strictly-inc}
        If $(A_n)_{n<\alpha}$ represents a path in $\mc{T}_\mc{C}$ with no backtracking, then $(A_n)_{n<\alpha}$ is strictly increasing.
    \end{lemma}
    \begin{proof}
        By Lemma \ref{no-backtrack}, we have $A_n\neq A_{n-1}^c$ for every $1\leq n<\alpha$. Thus, since $A_n\in\mc{U}_n=\mc{U}_{n-1}\cup\l\{A_{n-1}^c\r\}\comp\l\{A_{n-1}\r\}$, we see that $A_n\in\mc{U}_{n-1}$. Clearly $A_n\neq A_{n-1}$. It suffices to remove the three cases when $A_{n-1}\subseteq A_n^c$, $A_n\subseteq A_{n-1}$, and $A_n^c\subseteq A_{n-1}$, since then nestedness of $\mc{C}$ gives us $A_{n-1}\subset A_n$, as desired.
        \begin{itemize}
            \item If $A_{n-1}\subseteq A_n^c$, then $A_n^c\in\mc{U}_{n-1}$ by upward-closure of $\mc{U}_{n-1}$, a contradiction.
            \item If $A_n\subseteq A_{n-1}$, then $A_{n-1}\in\mc{U}_n$, contradicting the definition of $\mc{U}_n$.
            \item If $A_n^c\subseteq A_{n-1}$, then $A_{n-1}\in\mc{U}_{n+1}$ by upward-closure of $\mc{U}_{n+1}\ni A_n^c$. But since $A_{n-1}\neq A_n^c$, we have by definition of $\mc{U}_{n+1}$ that $A_{n-1}\in\mc{U}_n$, a contradiction.\qed
        \end{itemize}
    \end{proof}

    \begin{proposition}\label{acyclic}
        $\mc{T}_\mc{C}$ is acyclic.
    \end{proposition}
    \begin{proof}
        Let $(\mc{U}_i)_{i\leq n}$ be a cycle in $\mc{T}_\mc{C}$, say represented by $(A_i)_{i\leq n}$. Then $A_0=A_n$, a contradiction since $(A_i)_{i\leq n}$ is strictly-increasing by Lemma \ref{strictly-inc}.
    \end{proof}

    \begin{proposition}\label{tree}
        
    \end{proposition}
    \begin{proof}
        
    \end{proof}
\end{document}
