\documentclass{amsart}
\usepackage{amsfonts, amsmath, amssymb, amsthm, amsrefs}
\usepackage{tikz, tikz-3dplot, tikzpagenodes}
\usepackage{graphicx, xcolor, geometry, mdframed}
\usepackage{enumitem, nicematrix}
\usepackage{mathtools, mathrsfs}
\usepackage{caption, subcaption}
\usepackage{bookmark, xifthen}
\usepackage{transparent}

\usetikzlibrary{matrix, positioning, patterns, decorations.markings, arrows, arrows.meta, backgrounds, math, cd}
\tikzset{->-/.style={decoration={ markings, mark=at position #1 with {\arrow{>}}},postaction={decorate}}}

\definecolor{darkBlue}{RGB}{0, 0, 138}
\hypersetup{colorlinks=true, allcolors=darkBlue}
\theoremstyle{definition}\newtheorem{question}{\color{red}Question}

\input{macros.sty}

\begin{document}
    \title{Nested Collection of Cuts}
    \author{Zhaoshen Zhai}
    \maketitle

    \begin{notation}
        Let $X$ be a set. For any family $\mc{C}\subseteq2^X$ closed under the complement operation $\lnot:2^X\to2^X$, we let $\mc{C}^\ast\coloneqq\mc{C}\comp\l\{\em,X\r\}$ denote the \textit{non-trivial} elements in $\mc{C}$.
    \end{notation}

    \section{Preliminaries}

    \subsection{Ultrafilters and their bases}

    We recall some standard notions, mainly to establish notation.

    \begin{definition}
        A \textit{filter} on $X$ is a collection $\mc{F}\subseteq\pow^\ast(X)$ that is closed-upwards and closed under pairwise intersections, i.e.
        \begin{enumerate}
            \item If $A\in\mc{F}$ and $B\supseteq A$, then $B\in\mc{F}$; and
            \item If $A_1,A_2\in\mc{F}$, then $A_1\cap A_2\in\mc{F}$.
        \end{enumerate}
    \end{definition}

    \begin{lemma}
        For any family $\mc{F}_0\subseteq\pow^\ast(X)$, the following are equivalent.
        \begin{enumerate}
            \item For all $F_1,F_2\in\mc{F}_0$, there is some $F\in\mc{F}_0$ such that $F\subseteq F_1\cap F_2$.
            \item The collection $\mc{F}\coloneqq\l\{A\subseteq X\st A\supseteq F\textrm{ for some }F\in\mc{F}_0\r\}\subseteq\pow^\ast(X)$ is a filter on $X$.
        \end{enumerate}
    \end{lemma}
    \begin{proof}
        If (1) holds, then $\mc{F}$ is upward closed since if $A\supseteq F$ for some $F\in\mc{F}_0$, then $B\supseteq F$ for any $B\supseteq A$. Moreover, if $A_i\supseteq F_i$ for $i=1,2$, then $A_1\cap A_2\supseteq F_1\cap F_2\supseteq F$ for some $F\in\mc{F}_0$ furnished by (1).

        Conversely, let $F_1,F_2\in\mc{F}_0$. Then $F_1,F_2\in\mc{F}$, so $F_1\cap F_2\in\mc{F}$ by (2). Thus $F_1\cap F_2\supseteq F$ for some $F\in\mc{F}_0$, by definition of $\mc{F}$, as desired.
    \end{proof}

    \begin{definition}
        A \textit{basis for a filter} on $X$ is a family $\mc{F}_0\subseteq\pow^\ast(X)$ satisfying the equivalent definitions above, and we call the collection $\mc{F}$ in (2) the \textit{filter generated by $\mc{F}_0$}.
    \end{definition}

    \begin{definition}
        An \textit{ultrafilter} on $X$ is a $\subseteq$-maximal filter on $X$.
    \end{definition}

    \subsection{Nested cuts and $\mc{T}$}

    Throughout this subsection, let $(X,G)$ be a graph.

    \begin{definition}
        A \textit{cut} in $X$ is a subset $C\subseteq X$ contained in a single connected component $Y\subseteq X$ such that both $C$ and $Y\comp C$ are infinite but $\del_\mathsf{v}C$ is finite.
    \end{definition}
    \begin{definition}
        A family $\mc{C}$ of cuts in $X$ is \textit{nested} if every $C_1,C_2\in\mc{C}$ has an empty corner; i.e. $C_1^i\cap C_2^j=\em$ for some $i,j=\pm1$.
    \end{definition}

    \section{The graph $\mc{T}_\mc{C}$}

    \begin{theorem}
        Let $\mc{C}$ be a family of nested cuts in $X$. The graph $\mc{T}_\mc{C}$, whose
        \begin{itemize}
            \item Vertices are finitely-based ultrafilters on $\mc{C}$;
            \item Neighbors of
        \end{itemize}
        is acyclic. Furthermore, if {\color{red}{something}}, then $\mc{T}_\mc{C}$ is a tree.
    \end{theorem}
    \begin{proof}
        h
    \end{proof}

    \section{What goes wrong if $\mc{C}$ is non-nested?}
\end{document}
