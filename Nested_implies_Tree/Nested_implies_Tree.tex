\documentclass{amsart}
\usepackage{amsfonts, amsmath, amssymb, amsthm, amsrefs}
\usepackage{tikz, tikz-3dplot, tikzpagenodes}
\usepackage{graphicx, xcolor, geometry, mdframed}
\usepackage{enumitem, nicematrix}
\usepackage{mathtools, mathrsfs}
\usepackage{caption, subcaption}
\usepackage{bookmark, xifthen}
\usepackage{transparent}

\usetikzlibrary{matrix, positioning, patterns, decorations.markings, arrows, arrows.meta, backgrounds, math, cd}
\tikzset{->-/.style={decoration={ markings, mark=at position #1 with {\arrow{>}}},postaction={decorate}}}

\definecolor{darkBlue}{RGB}{0, 0, 138}
\hypersetup{colorlinks=true, allcolors=darkBlue}
\theoremstyle{definition}\newtheorem{question}{\color{red}Question}

\input{macros.sty}

\begin{document}
    \title{Tree of Orientations on a Nested Collection of Cuts}
    \author{Zhaoshen Zhai}
    \maketitle

    \section{Introduction}
    Let $\mc{C}\subseteq2^X$ be a collection of non-empty subsets of a set $X$. With the definitions in Section \ref{prelim}, we prove the following
    \begin{theorem}\label{main}
        If $\mc{C}$ is nested, then the graph $\mc{T}_\mc{C}$, whose:
        \begin{itemize}
            \item Vertices are finitely-based orientations on $\mc{C}$; and whose
            \item Neighbors of $\mc{U}\in V(\mc{T}_\mc{C})$ are $\mc{U}\symdiff\l\{A,A^c\r\}$ for every minimal $A\in\mc{U}$ with $A^c\in\mc{C}$;
        \end{itemize}
        is acyclic. Furthermore, $\mc{T}_\mc{C}$ is a tree iff $\mc{C}$ is closed under complements.
    \end{theorem}

    In particular, this applies to when $(X,G)$ is a graph and $\mc{C}$ is a nested collection of cuts on $X$.

    \section{Preliminaries}\label{prelim}

    \subsection{Orientations}

    Since we do not assume that $\mc{C}$ is closed under complements, we slightly modify the definition of orientations, as follows.

    \begin{definition}
        An \textit{orientation} on $\mc{C}$ is a subset $\mc{U}\subseteq\mc{C}$ such that
        \begin{enumerate}
            \item[1.] \textit{(Upward-closure)}. If $A\in\mc{U}$ and $B\in\mc{C}$ contains $A$, then $B\in\mc{U}$.
            \item[2.] \textit{(Ultra)}. If $A,A^c\in\mc{C}$, then either $A\in\mc{U}$ or $A^c\in\mc{U}$, but not both.
        \end{enumerate}
    \end{definition}

    \begin{remark}
        This coincides with the standard definition when $\mc{C}$ is a subpocset of $2^X$.
    \end{remark}

    \begin{lemma}\label{tree-well-defined}
        If $\mc{U}\subseteq\mc{C}$ is an orientation, then for any $\subseteq$-minimal $A\in\mc{U}$ with $A^c\in\mc{C}$, so is $\mc{U}\symdiff\l\{A,A^c\r\}$.
    \end{lemma}
    \begin{proof}
        That $\mc{V}\coloneqq\mc{U}\symdiff\l\{A,A^c\r\}=\mc{U}\cup\l\{A^c\r\}\comp\l\{A\r\}$ is upward-closed follows from $\subseteq$-minimality of $A$. Now, if $B,B^c\in\mc{C}$ and $B^c\not\in\mc{V}$, we need to show that $B\in\mc{V}$.

        To this end, note that $B^c\not\in\mc{V}$ implies $A\neq B$ and either $B=A^c$ or $B^c\not\in\mc{U}$. The former case follows from $A^c\in\mc{V}$, and for the latter, we have $B\in\mc{U}\comp\l\{A\r\}$ since $\mc{U}$ is ultra.
    \end{proof}

    \begin{remark}\label{tree-no-loops}
        In the above notations, clearly $\mc{U}\neq\mc{U}\symdiff\l\{A,A^c\r\}$. Furthermore, for any other such orientation $\mc{U}'$ and $A'\in\mc{U}'$, that $\mc{U}=\mc{U}'$ and $\mc{U}\symdiff\l\{A,A^c\r\}=\mc{U}'\symdiff\l\{A',A'^c\r\}$ together imply $A=A'$.
    \end{remark}

    \begin{definition}
        A \textit{base} for an orientation $\mc{U}\subseteq\mc{C}$ is a $\subseteq$-minimal subset $\mc{B}\subseteq\mc{U}$ such that $\mc{U}=\,\,\uparrow\!\mc{B}$, where
        \begin{equation*}
            \uparrow\!\mc{B}\coloneqq\bigcup_{B\in\mc{B}}\uparrow\!B\coloneqq\bigcup_{B\in\mc{B}}\l\{A\in\mc{C}\st A\supseteq B\r\}.
        \end{equation*}
    \end{definition}

    \begin{definition}
        A collection $\mc{C}$ is said to be \textit{nested} if every $C_1,C_2\in\mc{C}$ has an empty corner, i.e., $C_1^i\cap C_2^j=\em$ for some $i,j\in\l\{1,-1\r\}$, where $C^i\coloneqq C$ if $i=1$ and $C^i\coloneqq C^c$ if $i=-1$.
    \end{definition}

    \begin{remark}
        If $\mc{C}$ is nested, then every $\subseteq$-minimal $B\in\mc{C}$ induces an orientation $\uparrow\!B\coloneqq\l\{A\in\mc{C}\st A\supseteq B\r\}$, called the \textit{principal} orientation. Indeed, $\uparrow\!B$ is clearly upward-closed, and if $A,A^c\in\mc{C}$, then, by $\subseteq$-minimality of $B$ and nestedness of $\mc{C}$, either $A\supseteq B$ or $A^c\supseteq B$ (but clearly not both).

        This construction generalizes to any collection $\mc{B}\subseteq\mc{C}$ with each $B\in\mc{B}$ being $\subseteq$-minimal, in that $\uparrow\!\mc{B}$ is an orientation on $\mc{C}$.
    \end{remark}

    \begin{definition}
        An orientation $\mc{U}\subseteq\mc{C}$ is said to be \textit{finitely-based} if it admits a finite base.
    \end{definition}

    \begin{remark}\label{tree-finitely-based}
        If $\mc{U}=\,\,\uparrow\!\l\{B_1,\dots,B_n\r\}$ is finitely-based, then for any $\subseteq$-minimal $A\in\mc{U}$ with $A^c\in\mc{C}$, so is the orientation $\mc{V}\coloneqq\mc{U}\symdiff\l\{A,A^c\r\}$. Indeed, $A=B_i$ for some $1\leq i\leq n$, and $\mc{V}=\,\,\uparrow\!(\l\{A\r\}\cup\l\{B_j\r\}_{j\neq i})$.
    \end{remark}

    \section{The graph $\mc{T}_\mc{C}$}

    Fix a nested collection of non-empty subsets of a set $X$. Using Lemma \ref{tree-well-defined} and Remarks \ref{tree-no-loops} and \ref{tree-finitely-based}, we construct a graph $\mc{T}_\mc{C}$ whose:
    \begin{itemize}
        \item \textit{Vertices} of $\mc{T}_\mc{C}$ are finitely-based orientations on $\mc{C}$.
        \item \textit{Neighbors} of $\mc{U}\in V(\mc{T}_\mc{C})$ are the finitely-based orientations $\mc{U}\symdiff\l\{A,A^c\r\}$ for every minimal $A\in\mc{U}$.
    \end{itemize}

    The goal of this section is to establish Theorem \ref{main}, stating that $\mc{T}_\mc{C}$ is acyclic (Proposition \ref{acyclic}), and furthermore, $\mc{T}_\mc{C}$ is a tree precisely when $\mc{C}$ is closed under complements (Proposition \ref{tree}).

    \subsection{Paths in $\mc{T}_\mc{C}$}

    To show that $\mc{T}_\mc{C}$ is acyclic, we characterize backtracking paths in $\mc{T}_\mc{C}$ as follows.

    \begin{definition}
        Fix $\mc{U}_0\in V(\mc{T}_\mc{C})$ and $\alpha\leq\omega$. A sequence $(A_n)_{n<\alpha}\subseteq\mc{C}$ is said to \textit{represent a path from $\mc{U}_0$} if $(\mc{U}_n)_{n\leq\alpha}$, defined by $\mc{U}_n\coloneqq\mc{U}_{n-1}\symdiff\l\{A_{n-1}\r\}$ for every $1\leq n\leq\alpha$, is a path in $\mc{T}_\mc{C}$ with each $A_n\in\mc{U}_n$.
    \end{definition}

    \begin{remark}
        Any path in $\mc{T}_\mc{C}$ is represented by its sequence of flipped basis elements.
    \end{remark}

    \begin{lemma}\label{no-backtrack}
        Let $\alpha\geq3$. A path in $\mc{T}_\mc{C}$ from $\mc{U}_0$ represented by $(A_n)_{n\leq\alpha}$ has no backtracking iff $A_n\neq A_{n-1}^c$ for every $1\leq n<\alpha$.
    \end{lemma}
    \begin{proof}
        Take $2\leq n\leq\alpha$. It suffices to show that $\mc{U}_{n-2}=\mc{U}_n$ iff $A_{n-1}=A_{n-2}^c$.
        \begin{itemize}
            \item[($\Rightarrow$).] We have by definition that $\mc{U}_n=\mc{U}_{n-2}\cup\l\{A_{n-1}^c,A_{n-2}^c\r\}\comp\l\{A_{n-1},A_{n-2}\r\}$, so since $A_{n-2}\in\mc{U}_{n-2}=\mc{U}_n$, we have $A_{n-2}=A_{n-1}^c$ as desired.
            \item[($\Leftarrow$).] Again by definition, by noting that the basis-flipping cancels out.\qed
        \end{itemize}
    \end{proof}

    \begin{lemma}\label{strictly-inc}
        If $(A_n)_{n<\alpha}$ represents a path in $\mc{T}_\mc{C}$ with no backtracking, then $(A_n)_{n<\alpha}$ is strictly increasing.
    \end{lemma}
    \begin{proof}
        By Lemma \ref{no-backtrack}, we have $A_n\neq A_{n-1}^c$ for every $1\leq n<\alpha$. Thus, since $A_n\in\mc{U}_n=\mc{U}_{n-1}\cup\l\{A_{n-1}^c\r\}\comp\l\{A_{n-1}\r\}$, we see that $A_n\in\mc{U}_{n-1}$. Clearly $A_n\neq A_{n-1}$. It suffices to remove the three cases when $A_{n-1}\subseteq A_n^c$, $A_n\subseteq A_{n-1}$, and $A_n^c\subseteq A_{n-1}$, since then nestedness of $\mc{C}$ gives us $A_{n-1}\subset A_n$, as desired.
        \begin{itemize}
            \item If $A_{n-1}\subseteq A_n^c$, then $A_n^c\in\mc{U}_{n-1}$ by upward-closure of $\mc{U}_{n-1}$, a contradiction.
            \item If $A_n\subseteq A_{n-1}$, then $A_{n-1}\in\mc{U}_n$, contradicting the definition of $\mc{U}_n$.
            \item If $A_n^c\subseteq A_{n-1}$, then $A_{n-1}\in\mc{U}_{n+1}$ by upward-closure of $\mc{U}_{n+1}\ni A_n^c$. But since $A_{n-1}\neq A_n^c$, we have by definition of $\mc{U}_{n+1}$ that $A_{n-1}\in\mc{U}_n$, a contradiction.\qed
        \end{itemize}
    \end{proof}

    \begin{proposition}\label{acyclic}
        $\mc{T}_\mc{C}$ is acyclic.
    \end{proposition}
    \begin{proof}
        Let $(\mc{U}_i)_{i\leq n}$ be a cycle in $\mc{T}_\mc{C}$, say represented by $(A_i)_{i\leq n}$. Then $A_0=A_n$, a contradiction since $(A_i)_{i\leq n}$ is strictly-increasing by Lemma \ref{strictly-inc}.
    \end{proof}

    \begin{proposition}\label{tree}
        
    \end{proposition}
    \begin{proof}
        
    \end{proof}
\end{document}
