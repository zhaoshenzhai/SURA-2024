\documentclass{amsart}
\usepackage{amsfonts, amsmath, amssymb, amsthm, amsrefs}
\usepackage{tikz, tikz-3dplot, tikzpagenodes}
\usepackage{graphicx, xcolor, geometry, mdframed}
\usepackage{enumitem, nicematrix}
\usepackage{mathtools, mathrsfs}
\usepackage{caption, subcaption}
\usepackage{bookmark, xifthen}
\usepackage{transparent}

\usetikzlibrary{matrix, positioning, patterns, decorations.markings, arrows, arrows.meta, backgrounds, math, cd}
\tikzset{->-/.style={decoration={ markings, mark=at position #1 with {\arrow{>}}},postaction={decorate}}}

\definecolor{darkBlue}{RGB}{0, 0, 138}
\hypersetup{colorlinks=true, allcolors=darkBlue}
\theoremstyle{definition}\newtheorem{question}{\color{red}Question}

\input{macros.sty}

\begin{document}
    \title{Nested Collection of Cuts}
    \author{Zhaoshen Zhai}
    \maketitle

    \section{Introduction}

    With the definitions in Section \ref{prelim}, we prove the following
    \begin{theorem}
        If $(X,G)$ is a graph with a nested collection $\mc{C}\subseteq2^X$ of cuts, then the graph $\mc{T}_\mc{C}$ whose:
        \begin{itemize}
            \item Vertices are finitely-based orientations on $\mc{C}$; and whose
            \item Neighbors of $\mc{U}\in\mc{T}_\mc{C}$ are $\mc{U}\symdiff\l\{A,A^c\r\}$ for every minimal $A\in\mc{U}$;
        \end{itemize}
        is acyclic. Furthermore, if $\mc{C}$ is closed under complements, then $\mc{T}_\mc{C}$ is a tree.
    \end{theorem}

    \section{Preliminaries}\label{prelim}

    \subsection{Orientations}

    Let $\mc{C}\subseteq2^X$ be a collection of non-empty subsets of a set $X$. Since we do not assume that $\mc{C}$ is closed under complements, we slightly modify the definition of orientations, as follows.

    \begin{definition}
        An \textit{orientation} on $\mc{C}$ is a subset $\mc{U}\subseteq\mc{C}$ such that
        \begin{enumerate}
            \item[1.] \textit{(Upward-closure)}. If $A\in\mc{U}$ and $B\in\mc{C}$ contains $A$, then $B\in\mc{U}$.
            \item[2.] \textit{(Ultra)}. If $A,A^c\in\mc{C}$, then either $A\in\mc{U}$ or $A^c\in\mc{U}$, but not both.
        \end{enumerate}
    \end{definition}

    \begin{lemma}\label{tree-well-define}
        If $\mc{U}\subseteq\mc{C}$ is an orientation, then for any $\subseteq$-minimal $A\in\mc{U}$ with $A^c\in\mc{C}$, so is $\mc{U}\symdiff\l\{A,A^c\r\}$.
    \end{lemma}
    \begin{proof}
        That $\mc{V}\coloneqq\mc{U}\symdiff\l\{A,A^c\r\}=\mc{U}\cup\l\{A^c\r\}\comp\l\{A\r\}$ is upward-closed follows from $\subseteq$-minimality of $A$. Now, if $B,B^c\in\mc{C}$ and $B^c\not\in\mc{V}$, we need to show that $B\in\mc{V}$.

        To this end, note that $B^c\not\in\mc{V}$ implies $A\neq B$ and either $B=A^c$ or $B^c\not\in\mc{U}$. The former case follows from $A^c\in\mc{V}$, and for the latter, we have $B\in\mc{U}\comp\l\{A\r\}$ since $\mc{U}$ is ultra.
    \end{proof}

    \begin{remark}\label{tree-no-loops}
        In the above notations, clearly $\mc{U}\neq\mc{U}\symdiff\l\{A,A^c\r\}$. Furthermore, for any other such orientation $\mc{U}'$ and $A'\in\mc{U}'$, that $\mc{U}=\mc{U}'$ and $\mc{U}\symdiff\l\{A,A^c\r\}=\mc{U}'\symdiff\l\{A',A'^c\r\}$ together imply $A=A'$.
    \end{remark}

    \begin{definition}
        A \textit{base} for an orientation $\mc{U}\subseteq\mc{C}$ is a $\subseteq$-minimal subset $\mc{B}\subseteq\mc{U}$ such that $\mc{U}=\,\,\uparrow\!\mc{B}$, where
        \begin{equation*}
            \uparrow\!\mc{B}\coloneqq\bigcup_{B\in\mc{B}}\uparrow\!B\coloneqq\bigcup_{B\in\mc{B}}\l\{A\in\mc{C}\st A\supseteq B\r\}.
        \end{equation*}
    \end{definition}

    \begin{definition}
        A collection $\mc{C}$ is said to be \textit{nested} if every $C_1,C_2\in\mc{C}$ has an empty corner, i.e., $C_1^i\cap C_2^j=\em$ for some $i,j=\pm1$, where $C^i\coloneqq C$ if $i=1$ and $C^i\coloneqq C^c$ if $i=-1$.
    \end{definition}

    \begin{remark}
        If $\mc{C}$ is nested, then every $\subseteq$-minimal $B\in\mc{C}$ induces an orientation $\uparrow\!B\coloneqq\l\{A\in\mc{C}\st A\supseteq B\r\}$, called the \textit{principal} orientation. Indeed, $\uparrow\!B$ is clearly upward -closed, and if $A,A^c\in\mc{C}$, then, by $\subseteq$-minimality of $B$ and nestedness of $\mc{C}$, either $A\supseteq B$ or $A^c\supseteq B$ (but clearly not both).

        This construction generalizes to any collection $\mc{B}\subseteq\mc{C}$ with each $B\in\mc{B}$ being $\subseteq$-minimal, in that $\uparrow\!\mc{B}$ is an orientation on $\mc{C}$.
    \end{remark}

    \begin{definition}
        An orientation $\mc{U}\subseteq\mc{C}$ is said to be \textit{finitely-based} if it admits a finite base.
    \end{definition}

    \subsection{Cuts in graphs}

    Let $(X,G)$ be a graph.

    \begin{definition}
        A \textit{cut} in $X$ is a subset $C\subseteq X$ contained in a single connected component $Y\subseteq X$ with $\del_\mathsf{v}C$ finite such that both $C$ and $Y\comp C$ are infinite.
    \end{definition}

    \section{The graph $\mc{T}_\mc{C}$}
\end{document}
